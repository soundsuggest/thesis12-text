\chapter{Introduction}\label{chapter:introduction}

% CONTENTS :
% 	Een situering van het onderwerp in een ruimere context. Dit kan, al naar gelang het onderwerp, vrij ver gaan: situering binnen het vakdomein, situering binnen de maatschappelijke evolutie, raakvlakken met andere disciplines,...
% 	Een beknopt historisch overzicht van de evolutie van het onderwerp.
% 	Een bespreking van bestaande oplossingen en systemen.
% 	De verklaring van de titel en dus ook definitie van de termen die gebruikt worden in de titel.
% 	De doelstellingen van de masterproef.
% 	Een overzicht van de verschillende hoofdstukken

\section{Music suggestions and explanation systems}\label{chapter:introduction:section:context}

In their essence, recommender systems can be seen as filters applied on a large data sets. Ever since computer engineers started to develop this kind of systems, a wide range of algorithms have been designed and implemented to compute item recommendations\cite{burke:2002, melville:2002:CCF:777092.777124, pazzani:2007:CRS:1768197.1768209, rajaraman:2012}; each of them with their own advantages and disadvantages.

One of these recommendation algorithms is called collaborative filtering. In short, in this algorithm the system looks for similar user profiles, and recommends items that these do not share yet to each other\cite{rajaraman:2012}. A problem that is typical of collaborative recommendation, is that the end user does not know how the recommendation was computed, which is also referred to as the '\emph{black box problem}'\index{black box problem}\cite{herlocker:2000}.

To solve this problem, some kind of explanation system can be built to provide an explanation for the reasoning to arrive at the results. An ambitious approach would be to explain each step of the recommendation algorithm, but this not always possible or desired. Other examples of how additional context can be provided for explainations are indicating which tracks in a user's music library are closely related to the given recommendations, giving the system's confidence in the accuracy of the suggestions, et cetera\cite{herlocker:2000}.

Over the course of the last decade a wide range of explanation systems have been implemented. Many of these also use visualizations to explore user and/or item relationships\cite{bostandjiev:2012, crnovrsanin:2011:VRN:2421953.2422013, faridani:2010:opinionspace, gou:2011:SIF:2016656.2016671, gretarsson:2010, odonovan:2008}.


\section{Thesis objective}\label{chapter:introduction:section:objective}

The goal of this thesis is to design, implement and evaluate visualization and interaction techniques that will allow the user to gain insight into the recommendation process as well as actively steer the process. The elaboration of this thesis consists out of a literature study on the topic of visualization of music suggestions, and secondly a similar application that is designed and implemented\cite{kuleuven:2008:t313}.

In the literature study we will investigate recommender systems and their rationale, different visualization techniques, and how users gain insight into visualization. We will design a new visual explanation system. To evaluate this application we will use the insight evaluation techniques developed by Chris North in \cite{north:2006}, as well as usability testing based on evaluation techniques listed by Erik Duval in his course on user interfaces. These techniques include the think aloud protocol and use of SUS questionnaires.

Additional evaluation criteria, identified by Tintarev and Masthoff in \cite{tintarev:2007:SER:1547550.1547664} are used to compare explanation systems for recommender systems. In this thesis we will try to measure the performance of the explanation system based on these criteria.


\section{The visual explanation system}\label{chapter:introduction:section:application}

The application created for this thesis is a page action Chrome Extension that injects \emph{HTML} and \emph{JavaScript} into the recommendations page of \emph{Last.fm} at \url{http://last.fm/home/recs}. The application makes use of several \emph{JavaScript} libraries, such as D3\footnote{A library using SVG, HTML and JavaScript\cite{bostock:2012:d3js}; available at: \url{http://d3js.org/}} and jQuery\footnote{Available at: \url{http://jquery.com/}}, as well as a specific JavaScript library by Felix Bruns\footnote{Available at: \url{https://github.com/fxb/javascript-last.fm-api}} to facilitate the usage of the Last.fm API\footnote{Available at: \url{http://www.last.fm/api}}.

The application can be found in the Google Chrome web store\footnote{The SoundSuggest application can be found at: \url{https://chrome.google.com/webstore/detail/soundsuggest/jimmblcjmmjjfaklclmohcnabndlidmb}}.


\section{Next chapters}\label{chapter:introduction:section:chapters}

The rest of this thesis text is organized as follows. Chapter \ref{chapter:literature_study} presents a literature study on recommender systems, visualization techniques, insight gaining and visual explanation systems. Chapter \ref{chapter:requirements} tries to identify the users and how the application can be used. Chapter \ref{chapter:prototype} describes the testing methodology and the different iterations. Chapter \ref{chapter:implementation} looks at the different technologies that were used to develop the application and the architecture of the application, and discusses some of the specifics of the implementation. Chapter \ref{chapter:conclusion} concludes the thesis text. It provides an interpretation of the application's evaluation results, further conclusions, and a reflection on future work and opportunities.
