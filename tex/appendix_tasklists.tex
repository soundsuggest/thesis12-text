\chapter{Task lists for the user tests}\label{appendix:tasklists}

\section{Task list 1: testing insight and usability}\label{appendix:tasklists:prototype1}

The user is given some context, i.e., the user knows he/she is using a recommender system to find new music and he/she has a number artists in his/her artist library.

\begin{table}
	\caption{Task 1.1: hypothesis generation, no interaction allowed.}
	\begin{tabular}{ p{80px} | p{350px} }
		\hline
		\textbf{Goal/Output}			& Getting an idea of the user's mental model about the visualization when he/she is not allowed to interact with it. \\
		\textbf{Inputs}						& The user has an account, and has built up some listening history. \\
		\textbf{Assumptions}			& The user is logged in. The data has loaded. \\
		\textbf{Steps}						& The user will try to get an overview of the displayed data. Through eye-movements he/she will explore the visualization. The user forms a hypothesis on the visualization rationale. \\
		\textbf{Estimated time} 	& $5$ to $10$ minutes. \\
		\textbf{Instructions}			&
		
		Answer the following questions without interacting with the visualization:
		\begin{enumerate}
			\item Describe what you see. Which visual elements stand out? Which general structures can be identified?
			\item What do you think the visualization does?
			\item Which the elements of the user interface, do you think allow interaction?
			\item What do you think will happen when you:
				\begin{itemize}
					\item hover over an node of the graph?
					\item hover over one of the users?
					\item click on an item?
					\item click on a user?
				\end{itemize}
		\end{enumerate}
		\\
		\hline
	\end{tabular}
	\label{table:task:t1}
\end{table}

\begin{table}
	\caption{Task 1.2: Further familiarization, interaction allowed.}
	\begin{tabular}{ p{80px} | p{350px} }
		\hline
		\textbf{Goal/Output}			& Getting an idea of the user's mental model about the visualization. \\
		\textbf{Inputs}						& See table \ref{table:task:t1}. \\
		\textbf{Assumptions}			& See table \ref{table:task:t1}. \\
		\textbf{Steps}						& The user verifies his/her initial mental model through interactions with the visualization. If the initial mental model is not confirmed, it is adjusted. \\
		\textbf{Estimated time} 	& \\
		\textbf{Instructions}			&
		Try to interact with the visualization. Answer the following questions:
		\begin{enumerate}
			\item Which of the artists displayed in the graph are artist suggestions?
			\item What are the links or edges in the visualization?
			\item	Suppose you want to add an item to your profile, what steps would you undertake?
		\end{enumerate}
		\\
		\hline
	\end{tabular}
	\label{table:task:t2}
\end{table}


\begin{table}
	\caption{Task 1.3: Adding an artist to the music library and motivating the choice(s) made.}
	\begin{tabular}{ p{80px} | p{350px} }
		\hline
		\textbf{Goal/Output}			& \\
		\textbf{Inputs}						& \\
		\textbf{Assumptions}			& \\
		\textbf{Steps}						& \\
		\textbf{Estimated time} 	& \\
		\textbf{Instructions}			&
		Add an item to your profile. Answer the following questions:
		\begin{enumerate}
			\item Why did you choose that particular item?
			\item Can you give any other reasons why you should pick this item?
			\item Can you give reasons for choosing one of the other items?
		\end{enumerate}
		\\
		\hline
	\end{tabular}
	\label{table:task:t3}
\end{table}



\section{Task list 2: testing the first version of the settings menu}\label{appendix:tasklists:prototype2}

Tables \ref{table:task:t4}, \ref{table:task:t5}, and \ref{table:task:t6} give an overview of the tasks used in the user study to evaluate the settings menu.

\begin{table}
	\caption{Task 2.1: Change the number of shown recommendations up to $20$.}
	\begin{tabular}{ p{80px} | p{350px} }
		\hline
		\textbf{Goal/Output}			& The number of displayed recommendations in the graph has changed. \\
		\textbf{Inputs}						& \\
		\textbf{Assumptions}			& \\
		\textbf{Steps}						& \\
		\textbf{Estimated time} 	& \\
		\textbf{Instructions}			& \\
		\hline
	\end{tabular}
	\label{table:task:t4}
\end{table}


\begin{table}
	\caption{Task 2.2: Change the threshold to $0.3$.}
	\begin{tabular}{ p{80px} | p{350px} }
		\hline
		\textbf{Goal/Output}			& \\
		\textbf{Inputs}						& \\
		\textbf{Assumptions}			& \\
		\textbf{Steps}						& \\
		\textbf{Estimated time} 	& \\
		\textbf{Instructions}			& \\
		\hline
	\end{tabular}
	\label{table:task:t5}
\end{table}



\begin{table}
	\caption{Task 2.3: Change the colours to an encoding that you like.}
	\begin{tabular}{ p{80px} | p{350px} }
		\hline
		\textbf{Goal/Output}			& \\
		\textbf{Inputs}						& \\
		\textbf{Assumptions}			& \\
		\textbf{Steps}						& \\
		\textbf{Estimated time} 	& \\
		\textbf{Instructions}			& \\
		\hline
	\end{tabular}
	\label{table:task:t6}
\end{table}







\section{Task list 3: testing the performance of the evaluation system}\label{appendix:tasklists:prototype3}


\begin{table}
	\caption{Task 3.1: Find three neighbours that are closely related to you, based on the visualization.}
	\begin{tabular}{ p{80px} | p{350px} }
		\hline
		\textbf{Goal/Output}			& The user can find three closely related neighbours and can give an adequate motivation for his/her choice. \\
		\textbf{Inputs}						& \\
		\textbf{Assumptions}			& \\
		\textbf{Steps}						& \\
		\textbf{Estimated time} 	& \\
		\textbf{Instructions}			&
		Find three neighbours that are closely related to you, based on the visualization. Explain why these are closer neighbours than others neighbours on the graph.
		\\
		\hline
	\end{tabular}
	\label{table:task:t7}
\end{table}


\begin{table}
	\caption{Task 3.2: Find three recommended artists you think are interesting.}
	\begin{tabular}{ p{80px} | p{350px} }
		\hline
		\textbf{Goal/Output}			& The user can find three interesting artist recommendations and give an adequate motivation for his/her choice. \\
		\textbf{Inputs}						& \\
		\textbf{Assumptions}			& \\
		\textbf{Steps}						& \\
		\textbf{Estimated time} 	& \\
		\textbf{Instructions}			&
		Find three recommended artists you think are interesting. Explain why these artists are more interesting than other recommendations shown in the graph.
		\\
		\hline
	\end{tabular}
	\label{table:task:t8}
\end{table}


\begin{table}
	\caption{Task 3.3: Explain the recommendation rationale (transparency).}
	\begin{tabular}{ p{80px} | p{350px} }
		\hline
		\textbf{Goal/Output}			& \\
		\textbf{Inputs}						& \\
		\textbf{Assumptions}			& \\
		\textbf{Steps}						& \\
		\textbf{Estimated time} 	& \\
		\textbf{Instructions}			&
		Explain the recommendation rationale. How do you think Last.fm's recommender system works?
		\\
		\hline
	\end{tabular}
	\label{table:task:t9}
\end{table}


\begin{table}
	\caption{Task 3.4: Find a suggestion for an artist you didn't know about.}
	\begin{tabular}{ p{80px} | p{350px} }
		\hline
		\textbf{Goal/Output}			& \\
		\textbf{Inputs}						& \\
		\textbf{Assumptions}			& \\
		\textbf{Steps}						& \\
		\textbf{Estimated time} 	& \\
		\textbf{Instructions}			&
		Find a suggestion for an artist you didn't know about.
		\begin{itemize}
			\item Would you like to check our this artist's profile and listen one or more songs by this artist (persuasion)?
			\item Do you think the recommender system has made a good suggestion? Would you add it your profile (effectiveness)?
			\item How does it affect your trust in the recommender system (trust)?
		\end{itemize}
		\\
		\hline
	\end{tabular}
	\label{table:task:t10}
\end{table}


