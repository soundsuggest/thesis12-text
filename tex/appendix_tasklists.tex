\chapter{Task lists for the user tests}\label{appendix:tasklists}

\section{Task list 1: testing insight and usability}\label{appendix:tasklists:prototype1}

First the user is given some context, i.e., the user knows he/she is using a recommender system to find new music and he/she has a number artists in his/her artist library. Next tasks \ref{table:task:t1}, \ref{table:task:t2} and \ref{table:task:t3} are performed.

\begin{table}
	\caption{Task 1.1: hypothesis generation, no interaction allowed.}
	\begin{tabular}{ p{80px} | p{350px} }
		\hline
		\textbf{Goal/Output}			& Getting an idea of the user's mental model about the visualization when he/she is not allowed to interact with it. \\
		\textbf{Inputs}						& The user has an account, and has built up some listening history. \\
		\textbf{Assumptions}			& The user is logged in. The data has loaded. \\
		\textbf{Steps}						& The user will try to get an overview of the displayed data. Through eye-movements he/she will explore the visualization. The user forms a hypothesis on the visualization rationale. \\
		\textbf{Estimated time} 	& $5$ to $10$ minutes. \\
		\textbf{Instructions}			&
		
		Answer the following questions without interacting with the visualization:
		\begin{enumerate}
			\item Describe what you see. Which visual elements stand out? Which general structures can be identified?
			\item What do you think the visualization does?
			\item Which the elements of the user interface, do you think allow interaction?
			\item What do you think will happen when you:
				\begin{itemize}
					\item hover over an node of the graph?
					\item hover over one of the users?
					\item click on an item?
					\item click on a user?
				\end{itemize}
		\end{enumerate}
		\\
		\hline
	\end{tabular}
	\label{table:task:t1}
\end{table}

\begin{table}
	\caption{Task 1.2: Further familiarization, interaction allowed.}
	\begin{tabular}{ p{80px} | p{350px} }
		\hline
		\textbf{Goal/Output}			& Getting an idea of the user's mental model about the visualization. \\
		\textbf{Inputs}						& See table \ref{table:task:t1}. \\
		\textbf{Assumptions}			& See table \ref{table:task:t1}. \\
		\textbf{Steps}						& The user verifies his/her initial mental model through interactions with the visualization. If the initial mental model is not confirmed, it is adjusted. \\
		\textbf{Estimated time} 	& \\
		\textbf{Instructions}			&
		Try to interact with the visualization. Answer the following questions:
		\begin{enumerate}
			\item Which of the artists displayed in the graph are artist suggestions?
			\item What are the links or edges in the visualization?
			\item	Suppose you want to add an item to your profile, what steps would you undertake?
		\end{enumerate}
		\\
		\hline
	\end{tabular}
	\label{table:task:t2}
\end{table}


\begin{table}
	\caption{Task 1.3: Adding an artist to the music library and motivating the choice(s) made.}
	\begin{tabular}{ p{80px} | p{350px} }
		\hline
		\textbf{Goal/Output}			& \\
		\textbf{Inputs}						& See table \ref{table:task:t1}. \\
		\textbf{Assumptions}			& See table \ref{table:task:t1}. \\
		\textbf{Steps}						& The user clicks an artist of his/her choice. The user clicks \emph{Add to library} and confirms his/her action. The item is added to the profile and the visualization refreshes. \\
		\textbf{Estimated time} 	& $1$ to $5$ minutes. \\
		\textbf{Instructions}			&
		Add an item to your profile. Answer the following questions:
		\begin{enumerate}
			\item Why did you choose that particular item?
			\item Can you give any other reasons why you should pick this item?
			\item Can you give reasons for choosing one of the other items?
		\end{enumerate}
		\\
		\hline
	\end{tabular}
	\label{table:task:t3}
\end{table}



\section{Task list 2: testing the first version of the settings menu}\label{appendix:tasklists:prototype2}

Tables \ref{table:task:t4}, \ref{table:task:t5}, and \ref{table:task:t6} give an overview of the tasks used in the user study to evaluate the settings menu.

\begin{table}
	\caption{Task 2.1: Change the number of shown recommendations up to $20$.}
	\begin{tabular}{ p{80px} | p{350px} }
		\hline
		\textbf{Goal/Output}			& The number of displayed recommendations in the graph has changed. \\
		\textbf{Inputs}						& The user has a \emph{Last.fm} account. The user has authorized the application. \\
		\textbf{Assumptions}			& The user is logged in. The user has navigated to the recommendations page and the vsiualization has loaded. \\
		\textbf{Steps}						& Click the \emph{Settings} button and alter the slider for the number of recommendations. Click \emph{Save}. \\
		\textbf{Estimated time} 	& Less than a minute. \\
		\textbf{Instructions}			& Change the number of shown recommendations up to $20$. \\
		\hline
	\end{tabular}
	\label{table:task:t4}
\end{table}


\begin{table}
	\caption{Task 2.2: Change the threshold to $0.3$.}
	\begin{tabular}{ p{80px} | p{350px} }
		\hline
		\textbf{Goal/Output}			& The threshold used to cluster items has changed, which will affect the connectivity of the graph. \\
		\textbf{Inputs}						& See table \ref{table:task:t4}. \\
		\textbf{Assumptions}			& See table \ref{table:task:t4}. \\
		\textbf{Steps}						& Click the \emph{Settings} button and alter the slider for the threshold. Click \emph{Save}. \\
		\textbf{Estimated time} 	& Less than a minute. \\
		\textbf{Instructions}			& Change the threshold to $0.3$. \\
		\hline
	\end{tabular}
	\label{table:task:t5}
\end{table}



\begin{table}
	\caption{Task 2.3: Change the colours to an encoding that you like.}
	\begin{tabular}{ p{80px} | p{350px} }
		\hline
		\textbf{Goal/Output}			& The colour encodings for hover and click actions has changed, as well as the colour of the active user profile. \\
		\textbf{Inputs}						& See table \ref{table:task:t4}. \\
		\textbf{Assumptions}			& See table \ref{table:task:t4}. \\
		\textbf{Steps}						& Click the \emph{Settings} button and alter the the colour settings. Click \emph{Save}. \\
		\textbf{Estimated time} 	& About a minute or more. \\
		\textbf{Instructions}			& Change the colours to an encoding that you like. \\
		\hline
	\end{tabular}
	\label{table:task:t6}
\end{table}







\section{Task list 3: testing the performance of the evaluation system}\label{appendix:tasklists:prototype3}

Tasks \ref{table:task:t7}, \ref{table:task:t8} \ref{table:task:t9}, and \ref{table:task:t10} are used to further evaluate the explanation system properties.

\begin{table}
	\caption{Task 3.1: Find three neighbours that are closely related to you, based on the visualization.}
	\begin{tabular}{ p{80px} | p{350px} }
		\hline
		\textbf{Goal/Output}			& The user can find three closely related neighbours and can give an adequate motivation for his/her choice. \\
		\textbf{Inputs}						& See table \ref{table:task:t4}. \\
		\textbf{Assumptions}			& See table \ref{table:task:t4}. \\
		\textbf{Steps}						& Based on the visual thinking algorithm in table \ref{table:visual_thinking_algorithm}. \\
		\textbf{Estimated time} 	& $3$ minutes or more, depending on past experience. \\
		\textbf{Instructions}			&
		Find three neighbours that are closely related to you, based on the visualization. Explain why these are closer neighbours than others neighbours on the graph.
		\\
		\hline
	\end{tabular}
	\label{table:task:t7}
\end{table}


\begin{table}
	\caption{Task 3.2: Find three recommended artists you think are interesting.}
	\begin{tabular}{ p{80px} | p{350px} }
		\hline
		\textbf{Goal/Output}			& The user can find three interesting artist recommendations and give an adequate motivation for his/her choice. \\
		\textbf{Inputs}						& See table \ref{table:task:t4}. \\
		\textbf{Assumptions}			& See table \ref{table:task:t4}. \\
		\textbf{Steps}						& Based on the visual thinking algorithm in table \ref{table:visual_thinking_algorithm}. \\
		\textbf{Estimated time} 	& $3$ minutes or more, depending on past experience. \\
		\textbf{Instructions}			&
		Find three recommended artists you think are interesting. Explain why these artists are more interesting than other recommendations shown in the graph.
		\\
		\hline
	\end{tabular}
	\label{table:task:t8}
\end{table}


\begin{table}
	\caption{Task 3.3: Explain the recommendation rationale (transparency).}
	\begin{tabular}{ p{80px} | p{350px} }
		\hline
		\textbf{Goal/Output}			& The high level algorithm for collaborative filtering. \\
		\textbf{Inputs}						& See table \ref{table:task:t4}. \\
		\textbf{Assumptions}			& See table \ref{table:task:t4}. \\
		\textbf{Steps}						& Based on the visual thinking algorithm in table \ref{table:visual_thinking_algorithm}. \\
		\textbf{Estimated time} 	& $3$ minutes or more, depending on past experience. \\
		\textbf{Instructions}			&
		Explain the recommendation rationale. How do you think \emph{Last.fm}'s recommender system works?
		\\
		\hline
	\end{tabular}
	\label{table:task:t9}
\end{table}


\begin{table}
	\caption{Task 3.4: Find a suggestion for an artist you didn't know about.}
	\begin{tabular}{ p{80px} | p{350px} }
		\hline
		\textbf{Goal/Output}			& A new suggestion. \\
		\textbf{Inputs}						& See table \ref{table:task:t4}. \\
		\textbf{Assumptions}			& See table \ref{table:task:t4}. \\
		\textbf{Steps}						& The user looks at each of the suggestions and points out those that are new to him/her. \\
		\textbf{Estimated time} 	& About a minute to find new suggestions. Investigating interesting recommendations and answering the questions may take up up to $15$ minutes. \\
		\textbf{Instructions}			&
		Find a suggestion for an artist you didn't know about, and answer the following questions:
		\begin{itemize}
			\item Would you like to check our this artist's profile and listen one or more songs by this artist (persuasion)?
			\item Do you think the recommender system has made a good suggestion? Would you add it your profile (effectiveness)?
			\item How does it affect your trust in the recommender system (trust)?
		\end{itemize}
		\\
		\hline
	\end{tabular}
	\label{table:task:t10}
\end{table}


