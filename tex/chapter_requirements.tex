\chapter{Requirement analysis}\label{chapter:requirements}

%%%%%%%%%%%%%%%%%%%%%%%%%%%%%%%%%%%%%%%%%%%%%%%%%%%%%%%%%%%%%%%%%%%%%%%%%%%%%%%%%%%%%%%%%%%%%%%%%%%
%%%%%%%%%%%%%%%%%%%%%%											USER PROFILE											%%%%%%%%%%%%%%%%%%%%%
%%%%%%%%%%%%%%%%%%%%%%%%%%%%%%%%%%%%%%%%%%%%%%%%%%%%%%%%%%%%%%%%%%%%%%%%%%%%%%%%%%%%%%%%%%%%%%%%%%%
\section{User profile}

The target audience of the application includes users that look for new music or artists based on generated recommendations. Table \ref{tab:user_profile1} tries to establish a profile of the target users. Note that most of this user profile is what we expect the application's users to be like, rather than the result of surveys or other types of investigation.

\begin{table}[h]
\caption{User profile 1: sketching the targeted audience}
\begin{center}
	\begin{tabular}{ l p{300px} } % l = left-aligned column
		\hline
		\textbf{Skill set:}		& Has basic knowledge of computers; \\
													& Uses mouse for navigation; \\
													& Uses keyboard for entering text; \\
													& Is familiar with traditional website layouts; \\
													& Understands English; \\		
		\textbf{Behaviour:}		& Pays regular visits to sites like or similar to last.fm, imdb.com, netflix.com, youtube.com, and amazon.com and has an account on one or more of these websites; \\
													& Has a Facebook or other social media account; \\
													& Uses applications such as iTunes, Windows Media Player, and Spotify to listen to and purchase music; \\
													& Uses recommender systems to find new music, movies, books, et cetera. \\
		\textbf{Interests:}		& Music, videos, and other kinds of multimedia. \\
													& Online social networking. \\
		\textbf{Demography:}	& Aged between 18 and 30 years old; \\
													& Both male and female users; \\		
		\hline
	\end{tabular}
\end{center}
\label{tab:user_profile1}
\end{table}


User goals with a relevant a part of the application's functionality are the following:

\begin{itemize}
	\item The user wants suggestions, filtering out possibly interesting items from the vast item space. \textit{Suggestions are listed by the system, based on the user's interests. The user can add suggestions to his/her profile.}
	\item The user wants to gain insight into the reasoning behind the suggestions. \textit{Through the explanation system, the underlying conceptual model is visualized.}
	\item The user wants an indication of how reliable the suggestion is. \textit{By providing contextual information for each recommendation, the user can estimate how well the recommendation corresponds to his/her profile.}
\end{itemize}



%%%%%%%%%%%%%%%%%%%%%%%%%%%%%%%%%%%%%%%%%%%%%%%%%%%%%%%%%%%%%%%%%%%%%%%%%%%%%%%%%%%%%%%%%%%%%%%%%%%
%%%%%%%%%%%%%%%%%%%%%%											STORY BOARD 											%%%%%%%%%%%%%%%%%%%%%
%%%%%%%%%%%%%%%%%%%%%%%%%%%%%%%%%%%%%%%%%%%%%%%%%%%%%%%%%%%%%%%%%%%%%%%%%%%%%%%%%%%%%%%%%%%%%%%%%%%
\section{Story board}

The story board of the application is shown in figure \ref{figure:storyboard}.

%%%%%%%%%%%%%%%STORYBOARD
\begin{figure}
  \begin{center}
  \includegraphics[scale=0.7]{img/storyboard}
	\end{center}
  \caption{The story board for the SoundSuggest application.}
  \label{figure:storyboard}
\end{figure}



\section{User story}

\textit{Imagine you have a music library with a number of tracks in it. No doubt you will like certain tracks more than others. At a certain point you will want to expand your library. It is only natural that you will want to add music that is similar to the music you already like, but where should you begin to look for this kind of music? For this purpose you could use a recommender system.}

\textit{Let us assume you have plugged some recommender system into your music library and you have received a list of music suggestions. Which of these recommendations should you choose? Suppose you want to find the best ones first. Of course you could go through them all one by one, but that might take up quite some time. What it comes down to is that you don't know how the recommender system computed these recommendations, and as a result, you have a hard time making an educated decision where to start.}

\textit{Let's say that you have installed the the recommender system with an integrated explanation system. The explanation system visualizes how the items in your library are related to the recommendations, and provides additional statistics. Now, finding new, interesting music will hopefully become easier than ever before.}




%%%%%%%%%%%%%%%%%%%%%%%%%%%%%%%%%%%%%%%%%%%%%%%%%%%%%%%%%%%%%%%%%%%%%%%%%%%%%%%%%%%%%%%%%%%%%%%%%%%
%%%%%%%%%%%%%%%%%%%%%%											USE CASES													%%%%%%%%%%%%%%%%%%%%%
%%%%%%%%%%%%%%%%%%%%%%%%%%%%%%%%%%%%%%%%%%%%%%%%%%%%%%%%%%%%%%%%%%%%%%%%%%%%%%%%%%%%%%%%%%%%%%%%%%%
\section{Use case diagram}

Based on the discussion in section \ref{chapter:literature_study:section:user:subsection:insight}, four interactions can be identified: hovering of items, hovering of users, clicking of items, and clicking of users. The use case diagram is presented in Figure \ref{fig:use_case_diagram} lists each of these interactions. Tables \ref{tab:use_case1}, \ref{tab:use_case2}, \ref{tab:use_case3}, and \ref{tab:use_case4} in appendix \ref{appendix:use_cases} describe each use case in greater detail.

%%%%%%%%%%%%%%%USE CASE DIAGRAM
\begin{figure}
  \begin{center}
  \includegraphics[scale=0.7]{img/usecase_diagram}
	\end{center}
  \caption{Use case diagram of the SoundSuggest application.}
  \label{fig:use_case_diagram}
\end{figure}

%\FloatBarrier




