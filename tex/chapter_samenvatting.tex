\chapter*{Korte Samenvatting}\label{chapter:summary:dutch}
\addcontentsline{toc}{chapter}{Korte samenvatting}

%Finding new and interesting music in the abundant supply, is a difficult and complex problem. \emph{Recommender systems} address this by filtering out candidate suggestions from the item space based on a model of the user's taste\cite{song:2012}.

Het vinden van nieuwe, interessante muziek in het immense aanbod, is een lastige en tijdrovende zaak. Suggestiesystemen baseren zich op een model van de muziekvoorkeuren van de gebruiker bij het zoeken naar muzieksuggesties.


%Although many approaches exist to produce accurate recommendations, the rationale of recommender systems is often opaque towards the end user. This may cause decreased levels of acceptance of its recommendations. Herlocker et al. \cite{herlocker:2000} point out that \emph{explanation systems} can overcome this problem by providing insight into the reasoning behind suggestions.

Hoewel er vele strategie\"en bestaan om accurate suggesties te berekenen, is de eindgebruiker soms skeptisch tegenover de resultaten. Dit kan te wijten zijn aan het feit dat het systeem niet transparant is voor de gebruiker. Herlocker et al. \cite{herlocker:2000} leggen uit hoe dit probleem kan worden opgelost door middel van een uitlegsysteem.


%In this thesis we will look at a new explanation system for collaborative filtering, called \emph{SoundSuggest}. This system aims to explain  music recommendations made by \emph{Last.fm} using a graph-based approach giving an approximation of the utility matrix. The system is evaluated through a user study based on aims described by Tintarev and Masthoff\cite{tintarev:2007:SER:1547550.1547664} and properties of usability as listed by Nielsen\cite{nielsen:1993:UE:529793}. We will investigate the quality of insight gaining based on an evaluation method developed by Chris North\cite{north:2006}, and its effects on trust, effectiveness and persuasion of \emph{Last.fm} recommendations. Usability and related properties are evaluated through usability engineering and questionnaires.

In deze thesis stellen we \emph{SoundSuggest} voor, een uitlegsysteem voor de collaboratieve aanbeveler van \emph{Last.fm}. Deze applicatie visualizeert een benadering van de onderliggende utility matrix door middel van een op grafe gebaseerde aanpak.

De evaluatie van het systeem gebeurd door middel van een gebruikersstudie, gebaseerd op de doelstellingen voor uitlegsystemen van Tintarev en Masthoff\cite{tintarev:2007:SER:1547550.1547664}, en eigenschappen van gebruiksvriendelijkheid, opgelijst door Jakob Nielsen\cite{nielsen:1993:UE:529793}. De kwaliteit van inzicht wordt ge\"evalueerd door middel van een methode ontwikkeld door Chris North\cite{north:2006}. Ook het effect van inzicht op vertrouwen, effectiviteit, en overredingskracht wordt onderzocht. Gebruiksvriendelijkheid en gerelateerde problemen worden bepaald aan de hand van \emph{usability engineering} en \emph{system usability scale (SUS)} enqu\^etes.


%All of the test users in the study were able to describe the high level algorithm for collaborative recommendation. The test users were also able to apply gained insights to look for interesting recommendations. The average SUS score of the final iteration was $80.5$, suggesting the overall perceived usability is good.

Alle testgebruikers in de studie waren in staat om een hoogniveaubeschrijving te geven van het algoritme van collaboratieve filtering. Zij konden ook hun verworven inzichten toepassen bij het zoeken naar interessante suggesties. De gemiddelde SUS score bedroeg $80.5$ en geeft aan dat de algemene subjectieve gebruiksvriendelijkheid redelijk goed is.



