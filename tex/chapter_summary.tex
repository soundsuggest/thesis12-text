\chapter*{Abstract}\label{chapter:summary:english}
\addcontentsline{toc}{chapter}{Abstract}

% Korte samenvatting (max. 2 pagina's): Hierin dienen de belangrijkste doelstellingen en conclusies van de masterproef samengevat worden, in het Nederlands en in het Engels.

Finding new and interesting music in the abundant supply, is a difficult and complex problem. \emph{Recommender systems} address this by filtering out candidate suggestions from the item space based on a model of the user's taste\cite{song:2012}.

Although many approaches exist to produce accurate recommendations, the rationale of recommender systems is often opaque towards the end user. This may cause decreased levels of acceptance of its recommendations. Herlocker et al. \cite{herlocker:2000} point out that \emph{explanation systems} can overcome this problem by providing insight into the reasoning behind suggestions.

In this thesis we will look at a new explanation system for \emph{collaborative filtering}, called \emph{SoundSuggest}. This system aims to explain  music recommendations made by \emph{Last.fm} using a graph-based approach giving an approximation of the \emph{utility matrix}. The system is evaluated through a user study based on aims described by Tintarev and Masthoff\cite{tintarev:2007:SER:1547550.1547664} and properties of usability as listed by Jakob Nielsen\cite{nielsen:1993:UE:529793}. We will investigate the quality of insight gaining based on an evaluation method developed by Chris North\cite{north:2006}, and its effects on trust, effectiveness and persuasion of \emph{Last.fm} recommendations. Usability and related properties are evaluated through \emph{usability engineering} and \emph{system usability scale (SUS)} questionnaires.

All of the test users in the study were able to describe the high level algorithm for collaborative recommendation. The test users were also able to apply gained insights to look for interesting recommendations. The average SUS score of the final iteration was $80.5$, suggesting the overall perceived usability is good. %Remaining issues are related to scalability