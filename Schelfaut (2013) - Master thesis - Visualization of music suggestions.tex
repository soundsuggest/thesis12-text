% ----------- Cover Master Thesis Faculty of Sciences ---------------
% This document should be compiled with pdflatex.  If you want to use
% latex to compile to dvi/ps, you have to convert the images to (e)ps
%                           -- December 2012
% -------------------------------------------------------------------
\RequirePackage{fix-cm}
\documentclass[12pt,a4paper,oneside]{book}

% ------------------------- Load packages ---------------------------
% You can eventually add these while you load other packages
% in case you want to integrate the titlepage with the rest of your thesis
% -------------------------------------------------------------------
\usepackage{graphicx,xcolor,textpos}
\usepackage{helvet}
\usepackage{hyperref}
\usepackage{makeidx}
%\usepackage{subfigure} % DEPRECATED
\usepackage{caption}
\usepackage{subcaption}
\usepackage[]{algorithm2e}
\usepackage{longtable}
\usepackage{listings}
\usepackage{color}
\usepackage{rotating}
\usepackage{float}
\usepackage{placeins}
\usepackage{pdfpages}
\usepackage{nomencl}
\usepackage{tabularx}


%%%%%%%%%%%%%%%%%%%%%%%%%%%%%%%%%%%%%%%%%%%%
%% DEFINE JavaScript listings

\definecolor{lightgray}{rgb}{.9,.9,.9}
\definecolor{darkgray}{rgb}{.4,.4,.4}
\definecolor{purple}{rgb}{0.65, 0.12, 0.82}

\lstdefinelanguage{JavaScript}{
  keywords={typeof, new, true, false, catch, function, return, null, catch, switch, var, if, in, while, do, else, case, break},
  keywordstyle=\color{blue}\bfseries,
  ndkeywords={class, export, boolean, throw, implements, import, this},
  ndkeywordstyle=\color{darkgray}\bfseries,
  identifierstyle=\color{black},
  sensitive=false,
  comment=[l]{//},
  morecomment=[s]{/*}{*/},
  commentstyle=\color{purple}\ttfamily,
  stringstyle=\color{red}\ttfamily,
  morestring=[b]',
  morestring=[b]"
}

\lstset{
   language=JavaScript,
   backgroundcolor=\color{lightgray},
   extendedchars=true,
   basicstyle=\footnotesize\ttfamily,
   showstringspaces=false,
   showspaces=false,
   numbers=left,
   numberstyle=\footnotesize,
   numbersep=9pt,
   tabsize=2,
   breaklines=true,
   showtabs=false,
   captionpos=b
}

% ------------------------ Page settings -----------------------------
% If you change these, the cover layout will also change.  In that
% case you have to adjust the latter manually.
% --------------------------------------------------------------------

\topmargin -10mm
\textwidth 160truemm
\textheight 240truemm
\oddsidemargin 0mm
\evensidemargin 0mm

% ---------------------- textpos settings ----------------------------
% Some additional settings for the cover
% --------------------------------------------------------------------

\definecolor{green}{RGB}{172,196,0}
\definecolor{bluetitle}{RGB}{29,141,176}
\definecolor{blueaff}{RGB}{0,0,128}
\definecolor{blueline}{RGB}{82,189,236}
\setlength{\TPHorizModule}{1mm}
\setlength{\TPVertModule}{1mm}

% ---------------------------- index ---------------------------------
\makeindex
% ------------------------- nomenclature -----------------------------
\makenomenclature
% --------------------------- glossary -------------------------------
%\makeglossaries


\begin{document}

%%%%%%%%%%%%%%%%%%%%%%%%%%%%%%%%%%%%%%%%%%%%%%%%%%%%%%%%%%%%%%%%%
%%%%%%%%%%%%%%%%%%%%%%%%%%%%%%%%%%%%%%%%%%%%%%%%%%% COVER ENGLISH
%%%%%%%%%%%%%%%%%%%%%%%%%%%%%%%%%%%%%%%%%%%%%%%%%%%%%%%%%%%%%%%%%

% ----------------------- Cover --------------------------------------
% Please fill in:
% - The title and subtitle (if applicable)
%         to include a formula in the title or subtitle
%         use  \form{$...$}
% - Your name
% - Your (co)supervisor, mentor (if applicable)
% - Your master
% - The academic year
% --------------------------------------------------------------------
\thispagestyle{empty}
\newcommand{\form}[1]{\scalebox{1.087}{\boldmath{#1}}}
\sffamily
%
\begin{textblock}{191}(-24,-11)
	\colorbox{green}{\hspace{139mm}\ \parbox[c][18truemm]{52mm}{\textcolor{white}{FACULTY OF SCIENCE}}}
\end{textblock}
%
\begin{textblock}{70}(-18,-19)
	\textblockcolour{}
	\includegraphics*[height=19.8truemm]{LogoKULeuven}
\end{textblock}
%
\begin{textblock}{160}(-6,63)
	\textblockcolour{}
	\vspace{-\parskip}
	\flushleft
	\fontsize{40}{42}\selectfont \textcolor{bluetitle}{Visualization of music suggestions}\\[1.5mm]
	\fontsize{20}{22}\selectfont {A visual explanation system for collaborative filtering}
\end{textblock}

\begin{textblock}{160}(8,153)
\textblockcolour{}
\vspace{-\parskip}
\flushright
\fontsize{14}{16}\selectfont \textbf{Joris SCHELFAUT}
\end{textblock}
%
\begin{textblock}{90}(-6,181)
\textblockcolour{}
\vspace{-\parskip}
\flushleft
Supervisor: Prof. dr. ir. E. Duval\\[-2pt]
\textcolor{blueaff}{Affiliation \textsl{KU Leuven Department of Computer Science}}\\[5pt]
Co-supervisor: \textsl{Dr. J. Klerkx}\\[-2pt]
\textcolor{blueaff}{Affiliation \textsl{KU Leuven Department of Computer Science}}\\[5pt]
Co-supervisor: \textsl{Prof. dr. K. Verbert}\\[-2pt]
\textcolor{blueaff}{Affiliation \textsl{Technische Universiteit Eindhoven Department of Information Systems WSK\&I}}\\[5pt]
Mentor: \textsl{Dr. J. Klerkx}\\[-2pt]
Co-mentor: \textsl{Prof. dr. K. Verbert}\\[-2pt]
\end{textblock}
%
\begin{textblock}{160}(8,191)
\textblockcolour{}
\vspace{-\parskip}
\flushright
Thesis presented in\\[4.5pt]
fulfillment of the requirements\\[4.5pt]
for the degree of Master of Science\\[4.5pt]
in Applied Informatics\\
\end{textblock}
%
\begin{textblock}{160}(8,232)
\textblockcolour{}
\vspace{-\parskip}
\flushright
Academic year 2012-2013
\end{textblock}
%
\begin{textblock}{191}(-24,248)
{\color{blueline}\rule{550pt}{5.5pt}}
\end{textblock}
%
\vfill

\newpage

% Dissertation presented in fulfillment of the requirements for the degree of Master of Science in Applied Informatics

%%%%%%%%%%%%%%%%%%%%%%%%%%%%%%%%%%%%%%%%%%%%%%%%%%%%%%%%%%%%%%%
%%%%%%%%%%%%%%%%%%%%%%%%%%%%%%%%%%%%%%%%%%%%%%%%%%% COVER DUTCH
%%%%%%%%%%%%%%%%%%%%%%%%%%%%%%%%%%%%%%%%%%%%%%%%%%%%%%%%%%%%%%%

\thispagestyle{empty}
\newcommand{\form}[1]{\scalebox{1.087}{\boldmath{#1}}}
\sffamily
%
\begin{textblock}{191}(-24,-11)
	\colorbox{green}{\hspace{129mm}\ \parbox[c][18truemm]{62mm}{\textcolor{white}{FACULTEIT WETENSCHAPPEN}}}
\end{textblock}
%
\begin{textblock}{70}(-18,-19)
	\textblockcolour{}
	\includegraphics*[height=19.8truemm]{LogoKULeuven}
\end{textblock}
%
\begin{textblock}{160}(-6,63)
	\textblockcolour{}
	\vspace{-\parskip}
	\flushleft
	\fontsize{40}{42}\selectfont \textcolor{bluetitle}{Visualisatie van muzieksuggesties}\\[1.5mm]
	\fontsize{20}{22}\selectfont {Een visueel uitlegsysteem voor collaboratieve filtering}
\end{textblock}

\begin{textblock}{160}(8,153)
\textblockcolour{}
\vspace{-\parskip}
\flushright
\fontsize{14}{16}\selectfont \textbf{Joris SCHELFAUT}
\end{textblock}
%
\begin{textblock}{90}(-6,181)
\textblockcolour{}
\vspace{-\parskip}
\flushleft
Promotor: Prof. dr. ir. E. Duval\\[-2pt]
\textcolor{blueaff}{Affiliatie \textsl{KU Leuven Department of Computer Science}}\\[5pt]
Co-promotor: \textsl{Dr. J. Klerkx}\\[-2pt]
\textcolor{blueaff}{Affiliatie \textsl{KU Leuven Department of Computer Science}}\\[5pt]
Co-promotor: \textsl{Prof. dr. K. Verbert}\\[-2pt]
\textcolor{blueaff}{Affiliatie \textsl{Technische Universiteit Eindhoven Department of Information Systems WSK\&I}}\\[5pt]
Begeleider: \textsl{Dr. J. Klerkx}\\[-2pt]
Begeleider: \textsl{Prof. dr. K. Verbert}\\[-2pt]
\end{textblock}
%
\begin{textblock}{160}(8,191)
\textblockcolour{}
\vspace{-\parskip}
\flushright
Proefschrift ingediend\\[4.5pt]
tot het behalen van\\[4.5pt]
de graad van Master of Science\\[4.5pt]
in de Toegepaste Informatica\\
\end{textblock}
%
\begin{textblock}{160}(8,232)
\textblockcolour{}
\vspace{-\parskip}
\flushright
Academiejaar 2012-2013
\end{textblock}
%
\begin{textblock}{191}(-24,248)
{\color{blueline}\rule{550pt}{5.5pt}}
\end{textblock}
%
\vfill

\newpage

%%%%%%%%%%%%%%%%%%%%%%%%%%%%%%%




% In case you want to integrate the TeX-file for the titlepage
% with the rest of your thesis, you cab continue below
% ------------------------- First pages ---------------------------
% For table of contents, acknowlegments, ...
% -----------------------------------------------------------------
\rmfamily
\setcounter{page}{0}
\pagenumbering{roman}

\chapter*{Foreword}

% Voorwoord: Hierin moet het algemene doel van het werk samengevat worden. Verder dient iedereen die heeft bijgedragen hier bedankt te worden.



\chapter*{Summary}\label{chapter:summary:english}
\addcontentsline{toc}{chapter}{Summary}

% Korte samenvatting (max. 2 pagina's): Hierin dienen de belangrijkste doelstellingen en conclusies van de masterproef samengevat worden, in het Nederlands en in het Engels.

To find new and interesting music in the 

The rationale of recommender systems is often opaque towards the end user, possibly causing decreased levels of acceptance of its recommendations. Explanation systems can overcome this problem by providing insight into the reasoning behind suggestions\cite{herlocker:2000}.

In this thesis we will look at a white box model for collaborative filtering. This model is implemented as a visual explanation system called \emph{SoundSuggest} which aims to explain Last.fm's collaborative recommender. The system is evaluated through a user study. We will investigate the quality of insight gaining and its effects on trust, effectiveness and persuasion of Last.fm's recommendations.
\chapter*{Korte Samenvatting}\label{chapter:summary:dutch}
\addcontentsline{toc}{chapter}{Korte samenvatting}


\newpage
% Inhoud: Een overzicht van de inhoud moet ordelijk worden weergegeven met referentie naar de correctie paginanummers (maximum drie subniveaus). Alle pagina’s vóór hoofdstuk 1 dienen genummerd te worden met Romeinse cijfers (I, II, …). Vanaf hoofdstuk 1 gebeurt de nummering met Arabische cijfers (1, 2, …).
\tableofcontents

%% The List of Figures
%% -------------------
\clearpage
\addcontentsline{toc}{chapter}{List of Figures}
\listoffigures

%% The List of Tables
%% ------------------
\clearpage
\addcontentsline{toc}{chapter}{List of Tables}
\listoftables


% Lijst van afkortingen en lijst van symbolen: Hierin staan de belangrijkste afkortingen en symbolen die gebruikt worden in de masterproef, met hun betekenis en eenheid.
\newpage
%% List of acronyms
%% ----------------
%\printnomenclature
\chapter*{Abbreviations}\label{chapter:abbreviations}
\addcontentsline{toc}{chapter}{Abbreviations}

\begin{table}%
	\begin{tabular}{ l l}
	\textbf{AM}					&	Acoustic metadata							\\
	\textbf{CF}					& Collaborative filtering				\\
	\textbf{CBF}				& Content-based filtering				\\
	\textbf{CB}					& Content-based filtering				\\
	\textbf{EM}					&	Editorial metadata						\\
	\textbf{CM}					&	Cultural metadata							\\
	\textbf{CSS}				& Cascading stylesheet					\\
	\textbf{HTML}				& HyperText Markup Language 		\\
	\textbf{JSON}				& JavaScript Object Notation		\\
	\textbf{SVG} 				& Scalable vector graphics 			\\
	\end{tabular}
	%\caption{}
	\label{table:abbreviations}
\end{table}


\newpage
% -------------------------- Proper text --------------------------
% Introduction, chapters, ...
% -----------------------------------------------------------------
\setcounter{page}{0}
\pagenumbering{arabic}

%% TEX FILES (Chapters)
% Een uitgebreide inleiding die naast het schetsen van de aanpak en de gebruikte methodes het onderzoek in een bredere context plaatst.
\chapter{Introduction}\label{chapter:introduction}

\nomenclature{Fig.}{Figure}
\nomenclature{$A_i$}{Area of the $i^{th}$ component}

% CONTENTS :
% 	Een situering van het onderwerp in een ruimere context. Dit kan, al naar gelang het onderwerp, vrij ver gaan: situering binnen het vakdomein, situering binnen de maatschappelijke evolutie, raakvlakken met andere disciplines,...
% 	Een beknopt historisch overzicht van de evolutie van het onderwerp.
% 	Een bespreking van bestaande oplossingen en systemen.
% 	De verklaring van de titel en dus ook definitie van de termen die gebruikt worden in de titel.
% 	De doelstellingen van de masterproef.
% 	Een overzicht van de verschillende hoofdstukken

Music catalogues for online retail have become immense over the past decades. In 2013 the iTunes music catalogue was comprised of over 26 million tracks with users downloading over 25 billion songs\cite{itunes:2013:sales}. Today virtually anyone can create music and upload it to a music database such as \emph{bandcamp}\footnote{\url{https://bandcamp.com/}}, \emph{iTunes}\footnote{\url{http://www.apple.com/itunes/}}, or \emph{Last.fm}\footnote{\url{http://www.last.fm/}}\cite{bandcamp:2013:artists, itunes:2013:sales, lastfm:2012:home}. Well-known artists and tracks make up a very small portion of this item space, which is known as the \emph{Long-tail phenomenon}\cite{levy:2010}. As a result, finding new, interesting music has become a challenging task. \emph{Recommender systems} try alleviate this problem by filtering the item repository based on a user's music taste. Taste can be modelled by analyzing user preferences and tracking user behaviour, e.g., by analyzing a user's listening history\cite{song:2012}.

Ever since computer engineers started to develop this kind of systems, a wide range of algorithms have been designed and implemented to compute item recommendations\cite{burke:2002, melville:2002:CCF:777092.777124, pazzani:2007:CRS:1768197.1768209, rajaraman:2012}; each of them with their own advantages and disadvantages.

There are two commonly applied recommendation strategies\cite{rajaraman:2012}:

\begin{itemize}
	\item \textbf{Content-based filtering (CBF)\nomenclature{CBF}{Content-based filtering}}: Using chosen or modelled features of items to define similarity between items in the user profile and candidate suggestions;
	\item \textbf{Collaborative filtering (CF)\nomenclature{CF}{Collaborative filtering}}: Using overlap of item sets of each user profile to find possible suggestions in the difference of these item sets.
\end{itemize}

Although recommender systems have proven to be successful in terms of prediction accuracy, the success of a recommender system also relies on the trust in its recommendations by the end user. If the user does not know why a particular item is recommended to him, the user may be reluctant to check it out. Herlocker et al. \cite{herlocker:2000} describe this issue as the \emph{black box problem}. To improve acceptance of recommendations, they propose to build an explanation system presenting the user with a \emph{white box model} of the recommender system rationale.

There are different ways in which explanation systems can be designed. An ambitious approach would be to explain each step of the recommendation algorithm, but this not always possible or desired. Other examples of how additional context can be provided for explainations are indicating which tracks in a user's music library are closely related to the given recommendations, giving the system's confidence in the accuracy of the suggestions, et cetera\cite{herlocker:2000}.

Over the course of the last decade a wide range of explanation systems have been implemented. Many of these also use visualizations to explore user and/or item relationships\cite{bostandjiev:2012, gou:2011:SIF:2016656.2016671, gretarsson:2010, odonovan:2008, zhao:2010}.


\section{Thesis objective}\label{chapter:introduction:section:objective}

The initial thesis objectives as described in \cite{kuleuven:2008:t313} are two-fold:

\begin{enumerate}
	\item The conduction of a literature study on techniques for the visualization of music suggestions;
	\item The design, implementation and evaluation of an interactive visualization that will allow the user to gain insight into the recommendation process as well as actively steer the process.
\end{enumerate}

The literature study describes recommender systems and their rationale, different visualization techniques, how users gain insight into visualization, and a number of visual explanation systems. In this context we will compose a new white box model that can be used as an explanation system for the collaborative recommendation rationale. This initial design is tested and improved through a number of iterations, resulting in an application that satisfies a number of criteria. These criteria are based on a set of evaluation properties, as described in the next subsections.

\subsection{Evaluation properties}\label{introduction:objective:properties}

The explanation system will be evaluated based on seven aims described by Tintarev and Masthoff \cite{tintarev:2007:SER:1547550.1547664} listed in table \ref{table:explanation:aims}. Also learnability (Learn.) and memorability (Mem.), properties of usability as described by Nielsen\cite{nielsen:1993:UE:529793}, are evaluated.

\begin{table}
	\caption{Explanation aims. Table adapted from Tintarev and Masthoff \cite{tintarev:2007:SER:1547550.1547664}.}
	\begin{tabular}{ p{130px} | p{300px} }
		\hline
		\textbf{Aim} 						& \textbf{Definition} \\
		\hline
		\textit{Transparency} (Tra.)		&	Explain how the system works. \\
		\textit{Scrutability}	(Scr.)		&	Allow users to tell the system is wrong. \\
		\textit{Trust}									&	Increase users' confidence in the system. \\
		\textit{Effectiveness} (Efk.)		&	Help users make good decisions. \\
		\textit{Persuasiveness}	(Pers.)	&	Convince users to try or buy. \\
		\textit{Efficiency}	(Efc.)			& Help users make decisions faster. \\
		\textit{Satisfaction} (Sat.)		& Increase the ease of usability or enjoyment. \\
		\hline
	\end{tabular}
	\label{table:explanation:aims}
\end{table}

\subsection{Evaluation methodology}\label{introduction:objective:methodology}

Transparency is tested by evaluating insight into the recommendation process based on North's evaluation method. We will use the think aloud protocol to obtain observational data. In particular we are looking for a user to make "domain specific inferences and hypotheses"\cite{north:2006}.

Satisfaction, efficiency, and learnability are tested through think aloud usability testing and a summative \emph{system usabiliy scale} (SUS) questionnaire. SUS is a \emph{Linkert scale} method consisting out of 10 questions, listed in figure \ref{table:sus_questions}, to investigate the subjective usability of an application\cite{brooke:1996}. Memorability is tested by asking test users that participated in previous iterations to explain the recommender rationale again at the beginning of the test.

\begin{table}
	\caption{System usability scale questions.}
	\begin{tabular}{ p{20px} | p{410px} }
		\hline
		\texttt{Q1} 	&	I think that I would like to use this system frequently. \\
		\texttt{Q2}		&	I found the system unnecessarily complex. \\
		\texttt{Q3}		&	I thought the system was easy to use. \\
		\texttt{Q4} 	&	I think that I would need the support of a technical person to be able to use this system. \\
		\texttt{Q5}		&	I found the various functions in this system were well integrated. \\
		\texttt{Q6}		&	I thought there was too much inconsistency in this system. \\
		\texttt{Q7} 	&	I would imagine that most people would learn to use this system very quickly.  \\
		\texttt{Q8}		&	I found the system very cumbersome to use. \\
		\texttt{Q9}		& I felt very confident using the system. \\
		\texttt{Q10}	& I needed to learn a lot of things before I could get going with this system. \\
		\hline
	\end{tabular}
	\label{table:sus_questions}
\end{table}

Trust, persuasiveness, and effectiveness are evaluated through direct feedback from the test subjects.

% The level of trust in the recommender system can provide an additional metric ---> not used in literature

\subsection{Success criteria}\label{introduction:objective:criteria}

We aim to build a system that is accessible to non-expert users. To achieve this, we hope to achieve positive results in terms of overal usability, learnability, memorability and transparency.

% + explain why these properties are relevant for non-expert users



\section{The visual explanation system}\label{chapter:introduction:section:application}

The application created for this thesis is a page action Chrome Extension that injects \emph{HTML} and \emph{JavaScript} into the recommendations page of \emph{Last.fm} at \url{http://last.fm/home/recs}. The application makes use of several \emph{JavaScript} libraries, such as D3\footnote{A library using SVG, HTML and JavaScript\cite{bostock:2012:d3js}; available at: \url{http://d3js.org/}} and jQuery\footnote{Available at: \url{http://jquery.com/}}, as well as a specific JavaScript library by Felix Bruns\footnote{Available at: \url{https://github.com/fxb/javascript-last.fm-api}} to facilitate the usage of the Last.fm API\footnote{Available at: \url{http://www.last.fm/api}}.

Figure \ref{figure:soundsuggest} shows the application. The application can be found in the Google Chrome web store\footnote{The SoundSuggest application can be found at: \url{https://chrome.google.com/webstore/detail/soundsuggest/jimmblcjmmjjfaklclmohcnabndlidmb}}.

\begin{figure}
	\begin{center}
		\includegraphics[width=\columnwidth]{img/soundsuggest}%
	\end{center}
	\caption{The \emph{SoundSuggest} application.}
	\label{figure:soundsuggest}
\end{figure}


\section{Next chapters}\label{chapter:introduction:section:chapters}

The rest of this thesis text is organized as follows. Chapter \ref{chapter:literature_study} presents a literature study on recommender systems, visualization techniques, insight gaining and visual explanation systems. Chapter \ref{chapter:requirements} tries to identify the target audience and how the application can be used. Chapter \ref{chapter:prototype} describes the testing methodology and the different iterations. Chapter \ref{chapter:implementation} looks at the technologies that were used to develop the application and the architecture of the application, and discusses some of the specifics of the implementation. Chapter \ref{chapter:conclusion} concludes the thesis text. It provides an interpretation of the application's evaluation results, further conclusions, and a reflection on future work and opportunities.

\chapter{Literature study}\label{chapter:literature_study}

The term 'human-computer interaction' describes a phenomenon where two actors, the human or \emph{user} and the computer or \emph{system}, share a communication channel. The communication channel is a representation of data of the system towards the user. It is clear that both actors will impose certain restrictions on this visual communication channel\cite{shirley:2009, ware:2004}. In this chapter we will try to establish who these actors are, what this communication channel looks like, and what kind of restrictions are imposed upon the visual communication channel. 


\section{The user}\label{chapter:literature_study:section:user}

Who is the first actor? The answer to this question is of course very broad, so first we will try to list what we actually want to know. The goal is to allow the user to gain insight into the system through an interactive, visual explanation system. In conclusion, two specific questions arise:

\begin{itemize}
	\item how does a human gain insight?
	\item what kind of limitations are imposed by the user on the design of an interactive visualization?
\end{itemize}

The following subsections try to establish an answer to these questions. Most of the ideas in these subsections are drawn from a papers by Yi. et al. \cite{yi:2008}, North et al. \cite{north:2006}, Klein et al. \cite{Klein:2006:MSS:1158821.1159015, klein:2006:MSS:1175882.1176017}, and a book by Colin Ware \cite{ware:2004}.


\subsection{Insight gaining}\label{chapter:literature_study:section:user:subsection:insight}

What is insight\index{insight}? In \cite{north:2006} it is argued that insight is not a well-defined term. A formal definition might be too restrictive to capture its essence, and yet too broad to be useful. To quantify insight, \cite{north:2006} and \cite{yi:2008} list characteristics that allow a finer evaluation:

\begin{itemize}
	\item \textbf{Complex}: insight is complex in the sense that involves large amounts of data that form cognitive constructs, rather than individual units;
	\item \textbf{Deep}: insight is self-generating in a way, as insight provides a starting point for insight on the next level;
	\item \textbf{Qualitative}: insight is subjective, uncertain and can have multiple levels of resolution;
	\item \textbf{Unexpected}: insight is usually unpredictable, serendipitous and creative;
	\item \textbf{Relevant}: insight is deeply embedded in the data domain: it gives data meaning as it connects data to the existing domain knowledge;
\end{itemize}

The quality of insight can then be determined by quantifying each of these characteristics\cite{north:2006}. The previously described collection of properties defines insight. Now we will look at the closely related concept of sensemaking. The next paragraphs describe how a user can arrive at insight.


\subsubsection{Sensemaking}\label{chapter:literature_study:section:user:subsection:insight:subsubsection:sensemaking}

Sensemaking\index{sensemaking} plays an important part in insight gaining \cite{yi:2008}. The definitions for sensemaking may vary. We adapt the definition presented by \cite{Klein:2006:MSS:1158821.1159015} and \cite{yi:2008}.

In \cite{Klein:2006:MSS:1158821.1159015} sensemaking is looked at from a psychological perspective, a perspective of human-centered computing, and the perspective of naturalistic decision making. Sensemaking is then defined as follows: "sensemaking is a motivated, continuous effort to understand connections in order to anticipate their trajectories and act effectively"\cite{Klein:2006:MSS:1158821.1159015}.

Based on the dicussion in \cite{klein:2006:MSS:1175882.1176017} and \cite{yi:2008}, Soo Yi et al. describe the process of sensemaking. Sensemaking is a:

\begin{itemize}
	\item \textbf{Cyclic and iterative proceduce}: consisting out of a generation loop searching for representations, a data coverage loop instantiating the representations and finally shift representations;
	\item \textbf{Creation procedure}: being more about reasoning than discovery;
	\item \textbf{Retrospective procedure}: as people construct a framework and assign relevant information to a place withing this framework. If the data fits the framework well, the framework is confirmed, otherwise it may be updated or discarded;
\end{itemize}

An important remark made in \cite{Klein:2006:MSS:1158821.1159015} is that data fusion algorithms can reduce information overload, but they also pose challenges to sensemaking if the human can't form an accurate mental model of the machine, to understand why and how the algorithms are doing what they are doing.


\subsubsection{Processes of insight gaining}\label{chapter:literature_study:section:user:subsection:insight:subsubsection:processes}

Although sensemaking can play an important part in gaining insight, it is not the only path to arrive at insight \cite{yi:2008}. Soo Yi et al. \cite{yi:2008} identify four processes through which insight is established. Note that these processes are intertwined and often used together to generate insights. The processes are as follows\cite{yi:2008}:

\begin{itemize}
	\item \textbf{Provide overview}: in this process the individual gains understanding of the big picture of a dataset of interest. It allows the user to make a distinction between what is known to him/her and what is not;
	\item \textbf{Adjust}: in this process a person will explore a dataset by adjusting the level of abstraction and/or the range of selection. Typical actions involve filtering and grouping of data;
	\item \textbf{Detect pattern}: in this process the user will try to identify specific distributions, trends, frequencies, outliers or structure in the dataset;
	\item \textbf{Match mental model}: in this process the gap between data and cognitive model is bridged, reducing cognitive load and linking the present visual information with real-world knowledge.
\end{itemize}

The link with sensemaking is found in the cyclic and iterative nature of sensemaking - provide overview, adjust and detected pattern can be applied iteratively, as well as its creative and retrospective aspects - adjust and detect pattern create hypotheses and test them through various interaction techniques\cite{yi:2008}.


\subsubsection{Improving insight}\label{chapter:literature_study:section:user:subsection:insight:subsubsection:improving}

Yi et al. \cite{yi:2008} identify several ways in which the insight gaining process can be made more efficient. They list the system's interactivity, the quality of visual encodings and usability among others, as possible enablers for increased insight gaining. Naturally, improvident designs will act as barriers rather than enablers in the insight gaining process.

Interactivity of the system promotes the user's engagement into the dataset. Spending more time with the data will allow users to form more detailed and accurate hypotheses, and as a result greater insight\cite{yi:2008}. At the same time, while using the visualization, the user will become moe skilled at a task over time. Nonetheless, bare in mind that when performing long and tedious search tasks, vigilance will become an important aspect as well in the efficiency of data exploration\cite{ware:2004}.

Similarly visual encodings that are counter-intuitive will also increase the cognitive load. Other barriers on insight gaining are clutter, occlusion and data overload\cite{yi:2008}.

Usability is another aspect that may have an impact on the insight gaining process, as controls that are hard to use will inevitably occupy some of the cognitive capacity of the user\cite{yi:2008}. In the ISO standard ISO 9241-11, usability is defined as "the extent to which a product can be used by specified users to achieve specified goals with effectiveness, efficiency and satisfaction in a specified context of use"\cite{usabilitynet:2006:standards}.

Note that usability\index{usability} should not be considered a one-dimensional property of a user interface. Nielsen identifies several characteristics of usability in applications\cite{nielsen:1993:UE:529793}:

\begin{itemize}
	\item \textbf{Learnability}: if the system is easy to learn, the user can get started quickly;
	\item \textbf{Efficiency}: if the system is efficient to use, it will be possible to complete more work in less time;
	\item \textbf{Error rate and severity}: if the system should be robust and minimize faults;
	\item \textbf{Memorability}: once the system is learned, acquired skills should not be forgotten easily;
	\item \textbf{Satisfaction}: the system should be pleasant to use.
\end{itemize}


\subsection{Interactive visualization}\label{chapter:literature_study:section:user:subsection:interactive}

As we now have a better understanding of what insight is, we will try to establish how the insight gaining process works through visual data mining\index{visual data mining} and interactive visualization\index{visualization!interactive visualization}. The relation between insight gaining and data visualization has been pointed out in other research. Colin Ware \cite{ware:2004} describes interactive visualization as the interface between the user and the computer in a problem solving system. Keim \cite{keim:2002} notes that "idea behind visual data exploration, is to present data in a visual form, allowing the user to gain insight into the data, draw conclusions, and directly interact with the data".

In a chapter on visualization\index{visualization} in \cite{shirley:2009}, Tamara Munzner describes visualization as follows: "visualization allows the user to offload cognition to the perceptual system, using graphical data representations as a form of external memory. Therefore, by augmenting human capabilities, the data analyst is aided to understand, explore and form hypotheses of the data"\cite{shirley:2009}. In conclusion, the visual data exploration\index{visual data exploration|see{visual data mining}} process can then be understood as a hypothesis generation process\cite{keim:2002}.

In what follows, we try to describe how a human interacts with interactive visualization on a cognitive level. It will be clear that some parallels can be drawn with the insight gaining process. This should come as no surprise, since these processes are intertwined\cite{keim:2002, ware:2004, yi:2008}.

In \cite{ware:2004} interactive visualization is characterized by three classes of feedback loops:

\begin{itemize}
	\item \textbf{Data selection and manipulation loop}: the user selects and moves objects that are selected through simple interactions based on eye-hand coordination;
	\item \textbf{Exploration and manipulation loop}: the user tries to find his/her way through a large visual data space;
	\item \textbf{Problem solving loop}: the user forms hypotheses about the data and refines them through an augmented visualization process.
\end{itemize}

The following subsections describe each feedback loop in greater detail.


\subsubsection{Data selection and manipulation loop}\label{chapter:literature_study:section:user:subsection:interactive:subsubsection:loops1}

The quality of performance of selecting and manipulating data on a screen, depends on certain factors. Colin Ware discusses the following attributes:

\begin{itemize}
	\item \textbf{Reaction time}: This is the amount of time for a user to identify and select certain objects\cite{ware:2004}.
	\item \textbf{Types of interaction}: different types of interaction will have a different influence on user performance.
	\item \textbf{Learning}: The speed at which a user performs a task may decrease over time, as the user becomes more skilled at executing the task.
\end{itemize}

Each of these factors has been evaluated and since, they are captured in various laws. We will discuss some of them, as listed by Ware et al. in \cite{ware:2004}.

The reaction time is given by the \emph{Hick-Hyman law}\index{Hick-Hyman law}: $Time_{reaction} = a + b \log_{2}(C)$ with $C$ the number of choices and $a$ and $b$ empirically determined constants. This formula has been derived from experiments in which subjects had to press one of two buttons depending on the color of a light that was turned on or turned off\cite{ware:2004}.

The reaction-time may be influenced by many other factors such as the amount of visual noise, the distinctness of the signal and so on. If a person is allowed to make mistakes, the subject will respond faster, but at a cost of loss of accuracy\cite{ware:2004}. When performing long and tedious search tasks, vigilance will become an important aspect as well\cite{ware:2004}.

In \cite{ware:2004} different kinds of interactions are discussed such as selection, hover querries and path tracing. The time required for selecting an object in a two-dimensional space is determined by \emph{Fitt's law}\index{Fitt's law}: $Time_{selection} = a + b \log_{2}(D/W+1.0)$ with $D$ the distance to the target, $W$ the width of the target and $a$ and $b$ empirically determined constants\cite{ware:2004}.

A common way of selecting objects is through hover queries: the user drags a cursor over an object\cite{ware:2004}.

Another type of interaction is path tracing. The speed at which a user can trace or follow a given path is given by $v = W/\tau$ with velocity $v$, $W$ the path width and $\tau$ a constant depending on the motor control system of the user\cite{ware:2004}.

When a task is repeated over time, the user will become more efficient at performing this task. Depending on the difficulty of the task, this learning effect will be more or less prominent. The \emph{power law of practice}\index{power law of practice} describes this speed up \cite{ware:2004}: $\log(T_{n})=C-\alpha\log(n)$ in which $T_{n}$ is the time required to perform the task at the $n$-th trial, with $C=log(T_{1})$ and $\alpha$ the steepness of the learning curve.

Some actions are easier to learn than others. This is related to the \emph{stimulus-response (S-R) compatibility}\index{stimulus-response compatibility}. The stimulus-response compatibility refers to the way in which skills that have been learned before through everyday experience, are applied to computer control design. However it is not necessary to produce a whole virtual reality to create excellent computer interfaces. Research shows that mismatches between the interface and real world experience are not necessarily detrimental to the efficiency of human-computer interaction. Ware concludes that "it would be naive to conclude that computer interfaces should evolve toward VR simulations of real-world tasks (...). The magic of computers is that a single button click can often accomplish as much as a prolonged series of actions in the real world"\cite{ware:2004}.


\subsubsection{Exploration and manipulation loop}\label{chapter:literature_study:section:user:subsection:interactive:subsubsection:loops2}

The basic navigation control loop is described as an iterative process consisting out of the following steps\cite{ware:2004}:

\begin{itemize}
	\item On the human side, there is a logical and spatial model whereby the user understands the data space and his or her progress through it. If the data space is maintained for a long enough period of time, parts of the model may be encoded in the longterm memory;
	\item On the computer side, the visualization may be updated and refined from data mapped into the spatial model.
\end{itemize}



\subsubsection{Problem solving loop}\label{chapter:literature_study:section:user:subsection:interactive:subsubsection:loops3}



\section{The system}\label{chapter:literature_study:section:computer}

Before describing the visual communication channel between human and computer, it is necessary to know what exactly has to be explained through visualization. Therefore we will take a closer look at the system, i.e., the recommender system.

A recommender system\index{recommender system} is a system that computes item suggestions for users based on a ratings of related items in the user's profile/history. Additional information can be incorporated into the recommendation algorithm\index{recommendation algorithm} to refine suggestions. Several categorizations of these techniques are proposed in literature \cite{bostandjiev:2012, burke:2002, herlocker:2000, melville:2002:CCF:777092.777124, celma:2008:phd}.

One of the incentives behind creating recommender system is the 'long-tail phenomenon'\index{Long Tail}. This phenomenon can be explained as follows. Physical retail and warehouses can only keep a subset of all the available items in stock. These items are usually the most popular items on the market. Online vendors however, such as \emph{Amazon}\footnote{\url{http://www.amazon.com/}}, can offer a vastly larger subset of these items to clients, including also less popular and/or less known items\cite{rajaraman:2012}. Typically the long-tail phenomenon is visualized in a graph in which items are ordered by their popularity on the horizontal axis against the popularity rating on the vertical axis. Physical stores will offer only items in the first part of the graph, whereas the online vendors will also sell items from the remaining 'long tail' of the graph\cite{rajaraman:2012, celma:2008:phd}. Recommender systems then provide a means to find relevant items within this much larger range of items\cite{rajaraman:2012}. They enable to "connect supply and demand, introducing consumers to these new and newly available goods and driving demand down the tail "\cite{anderson:2006:LTW:1197299, celma:2008:phd}.

The success of recommender systems to achieve this objective has been the subject of some research, e.g. \cite{levy:2010} and \cite{celma:2008:phd}. Different classes of recommender algorithms favour different properties\cite{burke:2002, shani:2011:9780387858197}. This is elaborated further in subsection \ref{chapter:literature_study:section:computer:subsection:challenges}. For now its enough to see that the quality of recommendations is not uniquely defined. For example, in order to increase coverage of the item space, achieving high serendipity and novelty may be desired more than high recommendation accuracy, depending on the application context\cite{shani:2011:9780387858197, tripathi:2011, celma:2008:phd}.

Typical applications of recommender systems are product recommenders for online retailers, movie and music recommenders such as \emph{Netflix}\footnote{\url{http://www.netflix.com/}} and \emph{Last.fm}\footnote{\url{http://www.last.fm/}}, and news article recommenders in online news services\cite{levy:2010, rajaraman:2012, celma:2008:phd}.

Recommender system have opened up new possibilities in the landscape of online retail, and as a result, spurred the interest of businesses in this field. A remarkable initiative was the Netflix challenge\index{Netflix challenge}. In 2006, Netflix Inc. offered a prize to beat the performance of their recommendation algorithm by 10 percent. It gave a significant boost to the research on recommendation algorithms, and yielded a winning algorithm in September 2009\cite{bell:2007, rajaraman:2012}.


\subsection{Properties of recommendation algorithms}\label{chapter:literature_study:section:computer:subsection:properties}

in \cite{herlocker:2004:ECF:963770.963772} and \cite{shani:2011:9780387858197}, Herlocker et al. and Shani et al.  respectively, compare the performance of recommender systems. They list a number of metrics for recommendation algorithms. The resulting set of properties consist out of the following characteristics:

\begin{itemize}
	\item \textbf{Accuracy}: The accuracy of item recommendations. There are three broad classes of prediction accuracy measures:
	\begin{itemize}
		\item the prediction of the rating given by a user;
		\item the prediction whether or not a user will actually use the item (for example adding to a queue) opposed to predicting the rating itself;
		\item the prediction of a ranking among items rather than an explicit rating of each item independently.
	\end{itemize}
	\item \textbf{User preference}: The opinion of certain users may be more valuable than the opinion of others.
	\item \textbf{Coverage}: The proportion of items that the recommender system can recommend is referred to as catalog coverage. Another measure in this respect is the percentage of all items that are recommended to users. Finally we can also look at the diversity of the recommended items. Coverage can also mean the proportion of users or user interactions for which the system can recommend items.
The cold start problem relates to coverage as it measures the coverage for a specific type of users, namely new users.
	\item \textbf{Confidence}: Confidence in the recommendation can be defined as the systems trust in its recommendations or predictions. The most common measurement of confidence is the probability that the predicted value is indeed true, or the interval around the predicted value where predefined portion of the true values lie. Confidence bounds can be used to filter recommended items where the confidence in the predicted value is below some threshold.
	\item \textbf{Trust}: Trust refers to the user's trust in the system, as opposed to confidence.
	\item \textbf{Novelty}: Novel recommendations are recommendations for items that the user did not know about.
	\item \textbf{Serendipity}: Serendipity is a measure of how surprising the successful recommendations are. One can think of serendipity as the amount of relevant information that is new to the user in a recommendation, or alternatively as deviation from the 'natural' prediction.
	\item \textbf{Diversity}: Diversity is generally defined as the opposite of similarity. Note that an increase in diversity may correlate to a decrease in accuracy.
\end{itemize}


\subsection{A classification of recommendation algorithms}\label{chapter:literature_study:section:computer:subsection:algorithms}

Based on classifications presented in \cite{burke:2002} and \cite{celma:2008:phd}, a categorization of different types of recommendation strategies can be identified. We will only discuss the two most prominent ones, namely collaborative filtering (CF)\index{recommendation algorithm!collaborative filtering} and content-based filtering (CB)\index{recommendation algorithm!content-based filtering}\cite{herlocker:2000, rajaraman:2012}, and list some hybrid strategies. In the literature on recommender systems other general approaches that are commonly identified, are utility-based filtering, knowledge-based filtering, demographic filtering, and expert-based filtering\cite{burke:2002, bostandjiev:2012}.


\subsubsection{Collaborative recommendation}

Collaborative recommendation aggregates item ratings by users. By establishing overlaps between ratings in the corresponding user profiles, the system generates new item recommendations\cite{burke:2002, herlocker:2000}. A typical user profile in a collaborative system consists of a vector of items and their ratings, that continuously augmented as the user interacts with the system over time\cite{burke:2002}. 

For CF-based recommendation, there are two classes of entities: users $U$ and items $I$. The data itself can then be represented by a utility matrix\index{utility matrix} $A$. The entries $a_{i,j}$ of the utility matrix represent what is known about the degree of preference of user $u_{i}$ and item $i_{j}$\cite{rajaraman:2012}.

% Similarity / distance functions ...


\subsubsection{Content-based recommendation}

Content-based recommendation learns a profile of the user’s interests based on the features present in objects the user has rated. New recommendations can then be generated based on a similarity function on these features\cite{burke:2002, pazzani:2007:CRS:1768197.1768209}.

% Feature vectors ...


\subsection{Challenges for recommender systems}\label{chapter:literature_study:section:computer:subsection:challenges}

% Cold start / ramp up problem, first rater, new user, new item, gray sheep, profile trust (bad users), sparsity, black box

Each recommendation technique has benefits as well as drawbacks. Some of these apply to all or most types of recommendation strategies, while others are only relevant to certain cases.

%One of the challenges that 

% solutions: clustering, UV-decomposition, hybrid approaches, content-boosted / using estimations of empty entries

Both CF and CB-based recommendation algorithms suffer from the ramp-up\index{ramp-up|see{cold start}} problem in one way or the other. The 'ramp-up' or 'cold start' problem\index{cold start} (although they may refer to slightly different problems depending on the literature) is dual problem that encompasses two distinct, yet related problems as defined in \cite{burke:2002}:

\begin{itemize}
	\item \textbf{New User}\index{cold start!new user}: when a recommender system uses ratings by its users to compute item recommendations, it is hard to find neighbours for a user, who has a limited profile. As user profiles tend to build up over, new users usually fall in this category.
	\item \textbf{New Item}\index{cold start!new item}: a new item will most likely not have that many ratings associated with it, and as a result will not be easily recommended. This 'new item problem' typically emerges when new items are constantly added to the system; for example when browsing a constant stream of news articles. When new articles are introduced, not many users have had the chance yet to rate these items. In the case of a news feed, an additional problem is that these items are short-lived, meaning that at some point these item profiles will most likely stop receiving any ratings at all.
\end{itemize}

A problem that is typical of collaborative filtering is the 'gray sheep problem'\index{gray sheep}\cite{burke:2002, herlocker:2000}. The gray sheep problem occurs when a user falls between different clusters of users that may have contradicting item ratings. As a result, it is hard to determine how to classify the user\cite{burke:2002}.

Another issue with recommendation systems is that these system often appear as 'black boxes' towards the end user\index{black box}. The complexity of the algorithms used prevents the user from understanding the recommendation rationale\cite{zhao:2010}. This problem decreases the acceptance by the user of item suggestions. One of the solutions for this problem, proposed by Herlocker et al. in \cite{herlocker:2000}, is to provide an explanation system, i.e., the white box\index{white box}, on top of the recommender system that explains the recommendation process. This can be done through providing a transcript of the system's reasoning or through visualizations\cite{herlocker:2000}.


% more specifics on recommender properties



\section{The interface}\label{chapter:literature_study:section:interaction}

% What will we study?
% Link with previous sections?
In this section we will take a closer look at the visual communication channel.

Shirley et al.\cite{shirley:2009} lists three distinctive limitations:

\begin{itemize}
	\item \textbf{Computational capacity}: time complexity and memory usage of algorithms must allow a responsive user interface, especially in the case of interactive visualization.
	\item \textbf{Display capacity}: there is a trade-off between the benefits of maximizing the information density, i.e., the measure of the amount of encoded information against the amount of unused space, and causing visual overload.
	\item \textbf{Human perceptual and cognitive capacity}: optimizing the cognitive cost is one of the key aspects that make up a successful visualization, as visual and non-visual memory capacity are limited\cite{ware:2004}.
\end{itemize}

There has been done extensive research in this domain. Over the last two decades various new information visualization techniques have been developed. These techniques can be classified based on three criteria \cite{keim:2002}:

\begin{itemize}
	\item \textbf{Data}: a classification based on the structure and type of data;
	\item \textbf{Technique}: a classification based on characteristics of visualization techniques;
	\item \textbf{Interaction and distortion}: a classification based on the way in which interaction between user and visualization is enabled.
\end{itemize}


\subsection{Types of data}\label{chapter:literature_study:section:interaction:subsection:datatypes}

% context
% wat can be visualized, data types (recommender data)

Information visualization\index{information visualization}\index{infovis|see{information visualization}} has been focusing on on data sets that lack inherent spatial semantics, thus posing a challenge to map the abstract data onto a two-dimensional screen space\cite{keim:2002}.

There are different types of data and their characteristics will have an influence on the type of visualization. Tables of data consist out of rows, representing items, and columns, representing the data dimensions, or 'attributes'. The number of dimensions is referred to as the dimensionality\index{dimensionality} of the data set\cite{keim:2002}. There are three different kinds of dimensions, namely\cite{shirley:2009}:

\begin{itemize}
	\item \textbf{Quantitative}: numerical data on which arithmetic can be applied;
	\item \textbf{Ordered}: an enumeration that has a definite order;
	\item \textbf{Categorical}: data that has no specific ordering, and is distinguished by name only.
\end{itemize}

Relational data\index{relational data} on the other hand consists out of nodes\index{graph!node} and links or 'edges'\index{graph!edge}\cite{keim:2002, shirley:2009}. Both nodes and edges can have associated attributes.

In \cite{keim:2002} also text and hypertext, and algorithms and software are discussed as examples of other types of data. In the case of text and hypertext, standard visualizations are hard to use, as they cannot be described easily in terms of numbers. As a result, the data is first transformed into description vectors. Next, these vectors can be used in a visualization. Examples of software and algorithm visualization are flow diagrams, presentation using a graph-based structure of source code, and so on\cite{keim:2002}.


\subsection{Visual encoding and visual channels}\label{chapter:literature_study:section:interaction:subsection:encoding}

%Visual encoding principles

Visual encoding is defined as the mapping of data set attributes to a visual representation. The choice of visual encoding is one of the central problems in the visualization design\cite{shirley:2009}.

Visual encoding takes place through visual channels. A visual encoding corresponds to a graphical element, or ‘mark’. Examples of visual channels are spatial position, color, size, et cetera. The dimension of the mark may vary: a point is a zero-dimensional mark, a line a one-dimensional one, an area a two-dimensional one and so on.

A visual encoding has the following characteristics, as described in\cite{shirley:2009}:

\begin{itemize}
	\item \textbf{Distinguishability}: the ability of a user to distinguish between visual encodings;
	\item \textbf{Seperability}: Separable visual channels are opposed to integral visual channels, which are focused together on a pre-conscious level. Separable visual channels are safe to use for encoding multiple dimensions;
	\item \textbf{Pop-out}: selecting a channel and make it visually stands out from all the others.
\end{itemize}

%Visual channels

There is a variety of possible visual channels that a designer can turn to in order to create a visual encoding, such as color, spatial position, size, shape, orientation, and so on. The performance of the visual encoding (through a visual channel) depends on the type of data, i.e. quantitative, ordered or categorical \cite{shirley:2009}. Figure ... gives an overview of the performance for each category, adapted from \cite{shirley:2009}. Note that spatial position is the most accurate for each data type.


%Figure 1: Visual encoding performance for each data type


%\subsubsection{Colour}\label{chapter:literature_study:section:interaction:subsection:encoding:subsubsection:colour}

In \cite{shirley:2009} colour is considered in terms of three separate channels: hue, saturation and brightness. This allows for different encodings. Just like for most visual channels, the choice of the channel (hue, saturation or brightness) depends heavily on the type of data:

\begin{itemize}
	\item \textbf{Quantitative data}: uses a color map, a range of color values that can be continuous or discrete. It is recommended to use lightness instead of hue, as lightness has an implicit perceptual ordering. Moreover the human eye responds most to strong luminance. Hue on the other hand has a small range (around twelve values that can be reliably distinguished, including background and neutral colors);
	\item \textbf{Ordered data}: lightness and saturation are advised. As mentioned before, these have an implicit perceptual ordering;
	\item \textbf{Categorical data}: hue can be successfully applied for categorical data, keeping in mind its small range.
An important remark is that roughly 10\% of men is red-green color deficient. If a coding uses red and green, it may be wise to apply redundant coding using lightness or saturation in addition to hue \cite{shirley:2009}.
\end{itemize}

Spatial layouts form other visual channels. Although these tend to be the most accurate, spatial layouts in two and three dimensions have several weaknesses \cite{shirley:2009}:

\begin{itemize}
	\item \textbf{Occlusion}: parts of the data set become hidden by others. In the case of the mapping of abstract dimensions onto spatial positions, understanding the details of a three-dimensional visualization may be challenging, even if the user is allowed to change viewpoints;
	\item \textbf{Perspective distortion}: again, in the case of the mapping of abstract dimensions onto spatial positions, distances may convey meaning that may be distorted through perspective.
	\item \textbf{Text in arbitrary orientations}: special care has to be taken with text, as it may become very hard to read depending on the orientation.
\end{itemize}


\subsection{Visualization techniques}\label{}

% Techniques to overcome limitations of visual channels


\chapter{Requirement analysis}\label{chapter:requirements}

%%%%%%%%%%%%%%%%%%%%%%%%%%%%%%%%%%%%%%%%%%%%%%%%%%%%%%%%%%%%%%%%%%%%%%%%%%%%%%%%%%%%%%%%%%%%%%%%%%%
%%%%%%%%%%%%%%%%%%%%%%											USER PROFILE											%%%%%%%%%%%%%%%%%%%%%
%%%%%%%%%%%%%%%%%%%%%%%%%%%%%%%%%%%%%%%%%%%%%%%%%%%%%%%%%%%%%%%%%%%%%%%%%%%%%%%%%%%%%%%%%%%%%%%%%%%
\section{User profile}

The target audience of the application includes users that look for new music or artists based on generated recommendations. In \cite{song:2012}, a classification of music listeners is given. This classification by Jennings categorizes users, aged $16$-$45$, in one of four groups, as listed in table \ref{table:jennings:listeners}.


\begin{table}[h]
\caption{Categorization of music listeners by Jennings, adapted from \cite{song:2012}.}
\begin{center}
	\begin{tabular}{ l | r | p{250px} } % l = left-aligned column
		\hline
		\textbf{Type}			&		\textbf{Percentage}			&			\textbf{Features} \\
		\hline
		Savants 			& $7$		& Everything in life seems to be tied up with music. Their musical knowledge is extensive. \\
		Enthousiasts 	& $21$	& Music is a key part of life but is also balanced by other interests. \\
		Casuals				& $32$	& Music plays a welcome role, but other things are far more important. \\
		Indifferents	& $40$	& They would not loose much sleep if music ceased to exist, they are a predominant type of listeners of the whole population. \\
		\hline
	\end{tabular}
\end{center}
\label{table:jennings:listeners}
\end{table}

It is clear that indifferents are likely to have little interest in receiving particular artist recommendations, let alone finding out how the recommendations were computed. The focus of the application is mainly on enthousiasts and savants, as these users are more likely to look actively for music. These listeners are also more likely to look for music down the \emph{tail}\cite{song:2012}, cf. section \ref{chapter:literature_study:section:computer}.

Table \ref{tab:user_profile1} tries to establish a profile of the target users. Note that most of this user profile is what we expect the application's users to be like, rather than the result of surveys or other types of investigation.

\begin{table}[h]
\caption{User profile 1: sketching the targeted audience}
\begin{center}
	\begin{tabular}{ l p{300px} } % l = left-aligned column
		\hline
		\textbf{Skill set:}		&
		
		\begin{itemize}
			\item Has basic knowledge of computers;
			\item Uses mouse for navigation;
			\item Uses keyboard for entering text;
			\item Is familiar with traditional website layouts;
			\item Has basic proficiency in English;
		\end{itemize}
		 \\
		
		\textbf{Behaviour:}		&
		
		\begin{itemize}
			\item Pays regular visits to sites like or similar to \emph{Last.fm}, \emph{IMDb.com}, \emph{netflix.com}, \emph{YouTube.com}, and \emph{amazon.com} and has an account on one or more of these websites;
			\item Uses applications such as \emph{iTunes}, \emph{Windows Media Player}, and \emph{Spotify} to listen to and purchase music;
			\item Has used recommender systems before.
		\end{itemize}
		\\
		
		\textbf{Interests:}		& Can be classified as a music enthousiast or savant. \\
		\textbf{Demography:}	&
		\begin{itemize}
			\item Aged between 16 and 45 years old;
			\item Both male and female users.
		\end{itemize}
		\\
		
		\hline
	\end{tabular}
\end{center}
\label{tab:user_profile1}
\end{table}


User goals with a relevant a part of the application's functionality are the following:

\begin{itemize}
	\item The user wants suggestions, filtering out possibly interesting items from the vast item space. \textit{Suggestions are listed by the system, based on the user's interests. The user can add suggestions to his/her profile.}
	\item The user wants to gain insight into the reasoning behind the suggestions. \textit{Through the explanation system, the underlying conceptual model is visualized.}
	\item The user wants an indication of how reliable the suggestion is. \textit{By providing contextual information for each recommendation, the user can estimate how well the recommendation corresponds to his/her profile.}
\end{itemize}



%%%%%%%%%%%%%%%%%%%%%%%%%%%%%%%%%%%%%%%%%%%%%%%%%%%%%%%%%%%%%%%%%%%%%%%%%%%%%%%%%%%%%%%%%%%%%%%%%%%
%%%%%%%%%%%%%%%%%%%%%%											STORY (BOARD) 										%%%%%%%%%%%%%%%%%%%%%
%%%%%%%%%%%%%%%%%%%%%%%%%%%%%%%%%%%%%%%%%%%%%%%%%%%%%%%%%%%%%%%%%%%%%%%%%%%%%%%%%%%%%%%%%%%%%%%%%%%

\section{User story}

The following user story tries to establish a context in which the application might prove useful. It build on the target audience, defined earlier in table \ref{tab:user_profile1}.

\textit{Imagine you have a music library with a number of tracks in it. No doubt you will like certain tracks more than others. At a certain point you will want to expand your library. It is only natural that you will want to add music that is similar to the music you already like, but where should you begin to look for this kind of music? For this purpose you could use a recommender system.}

\textit{Let us assume you have plugged some recommender system into your music library and you have received a list of music suggestions. Which of these recommendations should you choose? Suppose you want to find the best ones first. Of course you could go through them all one by one, but that might take up quite some time. What it comes down to is that you don't know how the recommender system computed these recommendations, and as a result, you have a hard time making an educated decision where to start.}

\textit{Let's say that you have installed the the recommender system with an integrated explanation system. The explanation system visualizes how the items in your library are related to the recommendations, and provides additional statistics. Now, finding new, interesting music will hopefully become easier than ever before.}


\section{Story board}

The story board of the application is shown in figure \ref{figure:storyboard}. It further elaborates a particular use of the application.

%%%%%%%%%%%%%%%STORYBOARD
\begin{figure}
	\centering
	\begin{subfigure}[t]{0.4\textwidth}
					\centering
					\includegraphics[width=\textwidth]{img/storyboard01}
					\caption{}
					\label{figure:storyboard01}
	\end{subfigure}%
	~
	\begin{subfigure}[t]{0.4\textwidth}
					\centering
					\includegraphics[width=\textwidth]{img/storyboard02}
					\caption{}
					\label{figure:storyboard02}
	\end{subfigure}
	~
	\begin{subfigure}[t]{0.4\textwidth}
					\centering
					\includegraphics[width=\textwidth]{img/storyboard03}
					\caption{}
					\label{figure:storyboard03}
	\end{subfigure}
	~
	\begin{subfigure}[t]{0.4\textwidth}
					\centering
					\includegraphics[width=\textwidth]{img/storyboard04}
					\caption{}
					\label{figure:storyboard04}
	\end{subfigure}
	\caption{A selection of the screens used in the user study with paper prototype.}%
	\label{figure:storyboard}%
\end{figure}




%%%%%%%%%%%%%%%%%%%%%%%%%%%%%%%%%%%%%%%%%%%%%%%%%%%%%%%%%%%%%%%%%%%%%%%%%%%%%%%%%%%%%%%%%%%%%%%%%%%
%%%%%%%%%%%%%%%%%%%%%%											USE CASES													%%%%%%%%%%%%%%%%%%%%%
%%%%%%%%%%%%%%%%%%%%%%%%%%%%%%%%%%%%%%%%%%%%%%%%%%%%%%%%%%%%%%%%%%%%%%%%%%%%%%%%%%%%%%%%%%%%%%%%%%%
\section{Use case diagram}

Based on the discussion in section \ref{chapter:literature_study:section:user:subsection:insight}, four interactions can be identified: hovering of items, hovering of users, clicking of items, and clicking of users. The use case diagram is presented in Figure \ref{fig:use_case_diagram} lists each of these interactions. Tables \ref{tab:use_case1}, \ref{tab:use_case2}, \ref{tab:use_case3}, and \ref{tab:use_case4} in appendix \ref{appendix:use_cases} describe each use case in greater detail.

%%%%%%%%%%%%%%%USE CASE DIAGRAM
\begin{figure}
  \begin{center}
  \includegraphics[scale=0.7]{img/usecase_diagram}
	\end{center}
  \caption{Use case diagram of the \emph{SoundSuggest} application.}
  \label{fig:use_case_diagram}
\end{figure}

%\FloatBarrier





% Resultaten en bespreking van het onderzoek.
\chapter{Iterative development}\label{chapter:prototype}

% Shirley et al. : human-centered design process
%This iterative process can be human-centered design process described in \cite{shirley:2009}.

This chapter describes how the idea presented in chapter \ref{chapter:whitebox} is tested and improved through different development cycles. First the evaluation and development techniques used for the tests are described.


\section{Methodology}\label{chapter:prototype:section:methodology}

\subsection{Prototyping}\label{chapter:prototype:section:methodology:subsection:development}

% rapid prototyping

There are three types of prototypes that were used in the iterations:

\begin{itemize}
	\item Paper prototype;
	\item Digital prototype with 'fake' data or interaction effects;
	\item Digital prototype with working implementation.
\end{itemize}

It is obvious that for each category the resources that are required to build the prototype differ. The objective is to filter out most of the issues in the low cost designs, avoiding a greater cost in the more expensive prototypes.

% paper prototyping
%\subsubsection{Paper prototype}\label{chapter:prototype:s:methodology:ss:development:sss:paper}

\emph{Paper prototyping}\index{paper prototyping} is defined as "a variation of usability testing where representative users perform realistic tasks by interacting with a paper version of the interface that is manipulated by a person 'playing computer', who doesn’t explain how the interface is intended to work"\cite{snyder:2003:web}. It is a technique for designing, testing, and refining user interfaces \cite{snyder:2003}, and is closely related to usability testing \cite{snyder:2003:web}. In the last decade it has become a regularly applied technique in major businesses such as IBM, Digital, Honeywell, and Microsoft among others\cite{snyder:2003}.

%In \cite{usabilitynet:2006:paperprototyping} a number of benefits are associated with paper prototyping:
%\begin{itemize}
%	\item Usability problems can be already detected in the early stages of the design process, even before any code has been written.
%	\item The promotion of communication between designers and users.
%	\item Paper prototypes can be created and refined relatively easily, allowing for rapid design iterations.
%	\item Creating a paper prototype requires little resources.
%\end{itemize}


\subsection{Evaluation techniques}\label{chapter:prototype:section:methodology:subsection:evaluation}

The evaluation of an application prototype can be performed using one or more different techniques, and based on a range of varying criteria, such as: usability, usefulness, meaning, efficiency, accuracy and so on. Various techniques exist, such as questionnaires, usability engineering, expert evaluation, and usage tracking\cite{duval:2012:chi:evaluation}.

Methods may be \emph{formative}\index{formative!evaluation technique} or \emph{summative}\index{summative!evaluation technique}. Formative means that the evaluation occurs simultaneously with user task execution. Summative occurs after the user has performed all the required tasks\cite{duval:2012:chi:evaluation}.

The methods we will be using here are a variation on \emph{usability engineering}\index{usability engineering} called \emph{think aloud}\index{think aloud} user tests, and a special type of questionnaires called \emph{system usability scale} (SUS)\index{system usability scale} \emph{questionnaires}\index{questionnaire}.

In order to perform reliable usability tests, the test users have to be representative for the actual user population\cite{duval:2012:chi:evaluation}. The tasks that are being used, have to be representative of the system usage. Tasks also have to correspond to research questions to obtain relevant results\cite{snyder:2003}.

The number of users can often be limited to a certain amount. As the number of detectable problems is likely to be finite, from a certain point on adding more users to the usability test will not produce new or better results\cite{duval:2012:chi:evaluation, nielsen:2012:nngroup:diminishing_returns}. The graph in figure \ref{fig:nielsengraph}, adapted from \cite{nielsen:2012:nngroup:diminishing_returns}, illustrates this phenomenon. Nielsen argues that as a rule of tumb, five test users is enough to acquire reliable and valuable test results. Instead of doing one test with 15 users, use three iterations with 5 users each. Based on the graph, the first iteration will discover the majority of the usability problems; the next two tests will uncover the remaining 15\% of issues. Of course, this only holds on the condition that tasks performed by the users are similar for each iteration. Between each iteration, corrections are applied to the design\cite{nielsen:2012:nngroup:diminishing_returns}.

\begin{figure}
	\begin{center}
		\includegraphics[width=250px]{img/nielsen2012_usertests}
	\end{center}
	\caption{The curve shows the user test's diminishing returns beyond a certain amount of test users; adapted from \url{http://www.nngroup.com/articles/why-you-only-need-to-test-with-5-users/}.}
	\label{fig:nielsengraph}
\end{figure}


\subsubsection{Usability engineering}

In \cite{duval:2012:chi:evaluation}, two methods are described to perform usability engineering tests: usability labs, and think aloud\index{think aloud} testing. In a usability lab the user is observed while performing certain tasks. Data on task completion time, mouse clicks, eye-movement can be collected. Direct observation or cameras can be used to observe the user. To mimic real-life situations, also complete settings can be recreated in which the users would normally use the application\cite{duval:2012:chi:evaluation}.

Using usability labs can be rather costly, as labs need to be available and the required equipment may be expensive. The think aloud protocol is a variation on the usability lab method and is cheaper to perform, cf. 'discount usability engineering'\cite{duval:2012:chi:evaluation}. During a think aloud test, the user describes his/her reasoning for each action he/she undertakes\cite{nielsen:2012:nngroup:think_aloud}. This method has several advantages and disadvantages, as listed in table \ref{table:usability_engineering}, based on \cite{nielsen:2012:nngroup:think_aloud} and \cite{snyder:2003}.


\begin{table}%
	\begin{center}
		\begin{tabular}{l p{300px}}
			\hline
			Advantages		&		It is cheap to perform; \\
										&		It is robust; \\
										&		It is flexible; \\
										&		It is convincing; \\
										&		It is easy to learn; \\
			\hline
			Disadvantages	&		It creates an unnatural situation, as users usually don't say out loud everything they are about to do or think; \\
										&		The user may tend to filter his/her statements to avoid saying things that he/she may find silly or uninteresting; \\
										&		The facilitator may introduce bias in user behavior if he/she provides too much information when answering or instructing users; \\
			\hline
		\end{tabular}
	\end{center}
	\caption{Advantages and disadvantages of the think aloud protocol.}
	\label{table:usability_engineering}
\end{table}

\subsubsection{Questionnaires}

To obtain information other than observational data, the user is presented with a summative questionnaire. Questionnaires are used to obtain subjective information from the user about the user's experiences. Table \ref{table:questionnaires} lists several advantages and disadvantages of the use of questionnaires in usability studies, based on \cite{kirakowski:2013}.

\begin{table}%
	\begin{center}
		\begin{tabular}{l p{300px}}
			\hline
			Advantages		&		Evaluates the point of view of the user; \\
										&		Measures gained from a questionnaire are to a large extent, independent of the system, users, or tasks to which the questionnaire was applied; \\
										&		Quick and cost effective; \\
			\hline
			Disadvantages	&		Only the user's reaction as the user perceives the situation; \\
										&		Lack of detail, as questionnaires are usually designed to fit a number of different situations; \\
										&		Subjective data must be enhanced with performance, mental effort, and effectiveness data. \\
			\hline
		\end{tabular}
	\end{center}
	\caption{Advantages and disadvantages of the questionnaires.}
	\label{table:questionnaires}
\end{table}


There are several standardized questionnaires. The one used for the application evaluation is the system usability scale\index{system usability scale} (SUS). A system usability scale test is a questionnaire that consists out of ten specific questions that attempts to measure the user's perception of the application's usability. Each question is answered by checking one out of five checkboxes: checkbox one corresponds to strong disagreement with the statement, the fifth checkbox corresponds to strong agreement with the statement\cite{sauro:2011}. The ten questions are listed in appendix \ref{appendix:sus}.


\section{Story board}


\section{Iteration 1: paper prototype}\label{chapter:prototype:section:paper}

% 6. testen...
%		+ doel: wat wil je te weten komen?
%		+ methode: hoe ga je dat te weten komen?
%		+ rationale: welke andere manieren heb je overwogen en waarom niet weerhouden?
%		+ wie, wat, waar: n, demographics, logistics, ...
%		+ resultaten: wat heb je gemeten, geobserveerd? (zonder interpretatie)
%		+ besluit: wat ben je te weten gekomen? (link naar resultaten)
% 7. itereren...
%		+ doel: belangrijkste proble(e)m(en) dat/die je wil aanpakken
%		+ alternatieven: hoe kan je problemen aanpakken
%		+ keuze: en rationale
%		+ herneem die problemen bij volgende iteratie!



\section{Iteration 2: first digital prototype (SoundSuggest 1.x)}\label{chapter:prototype:section:soundsuggest1}




\section{Iteration 3: second digital prototype (SoundSuggest 2.x)}\label{chapter:prototype:section:soundsuggest2}




\section{Iteration 4: third digital prototype (SoundSuggest 3.x)}\label{chapter:prototype:section:soundsuggest3}






\chapter{Implementation: the SoundSuggest application}\label{chapter:implementation}

\section{Software design and application architecture}\label{chapter:implementation:section:design}


\section{Technologies}\label{chapter:implementation:section:technologies}

\subsection{The Last.fm API}\label{chapter:implementation:section:technologies:subsection:lastfm}



\subsection{D3.js JavaScript Library}\label{chapter:implementation:section:technologies:subsection:d3js}



\subsection{Chrome extensions}\label{chapter:implementation:section:technologies:subsection:chrome}



\section{Implementation}\label{chapter:implementation:section:implementation}

% Een conclusie, waarin een globale discussie van de onderzoeksresultaten wordt opgenomen. Eventueel kunnen hierin suggesties gedaan worden voor toekomstig onderzoek.
\chapter{Conclusion and future work}\label{chapter:conclusion}

% (1) de voornaamste bijdragen van de thesis proberen voor te stellen, (2) de doelstellingen herhalen en beschrijven in welke mate je deze doelstellingen gehaald hebt, (3) tekortkomingen van jouw werk en opportuniteiten voor toekomstig onderzoek en eventueel (4) een reflectie toevoegen: bijvoorbeeld als je opnieuw mocht beginnen wat zou je dan anders doen, wat heb je geleerd, etc.

% overview (geen subsection)

In the literature we discussed recommender systems and their general context. We listed system properties, described three common recommendation approaches, and listed typical issues and shortcomings of recommender systems. One of these issues is the black box problem for which the end user fails to gain insight into the recommendation process and as a result may have little trust in its recommendations. To solve this problem an explanation system can be used that explains the recommendation rationale.

In the next part of the litature study we looked at a way to visualize this rationale. We came up with a graph-based visualization representing the underlying utility matrix of collaborative recommendation, that uses Holten's edge-bundling algorithm along with node reduction to reduce the number of data dimensions, inspired by a visualization by Valdis Krebs.

Subsequently we looked at an evaluation method for visualization insight developed by Chris North. We also investigated the insight gaining process established by Klein et al. Finally we adapted Ware and Mitchell's visual thinking algorithm to describe how a user would interact with the visualization to solve a certain problem.

A number of visual explanation systems were discussed. To compare these systems we used a number of goals presented by Tintarev and Masthoff.

%Chapter \ref{chapter:requirements} investigated the target users, possibile scenarios of use and more elementary scenarios captured in use cases.

%In chapter \ref{chapter:prototype} we looked at a number of design and evaluation methods. The second part of this chapter described four different iterations in which a total of $15$ test users were observed while testing and evaluating the application. Between each iteration the detected usability issues were addressed and tested for success in the next. All the test users gained insight into the recommendation process and were able explain the recommendation rationale. All test users could give a reasoning for choosing certain recommendations based on the visualization. Deeper levels of insight were attained mainly, but not only, by users who had participated in more than one iteration.


\section{Objectives}

The first objective described in section \ref{chapter:introduction:section:objective} was to conduct a literature study on techniques for the visualization of music suggestions. This has not been entirely reflected in this text. Nonetheless an effort was made to link presented techniques either to the end product or other examples in the context of music recommendation.

%The literature study as it is, focuses more on  psychological aspects.
%In hindsight, 

The second objective was to design, implement and evaluate an interactive visualization that will allow the user to gain insight into the recommendation process as well as actively steer the process. The following success criteria for the application were listed in section \ref{introduction:objective:criteria} of the introduction:

\begin{itemize}
	\item Aimed at non-expert users with an average to high interest in music;
	\item Achieve high usability, in particular learnability and memorability;
	\item Provide transparency.
\end{itemize}

Although there were some casual listeners among the test users, the majority of participants was representative of the target audience. Results for the perceived usefulness in the last iteration vary between $2$ and $5$, suggesting that the first criteria has not been met entirely. Still, if the application would be developed further, and more users get tested, the average may still increase.

An overall average SUS score of $80.5$ in the final iteration suggests that the usability of the system is good, as perceived by users. However, the learnability of the system has perhaps some room for improvement.

Results indicate that our design can be effective in explaining the rationale of collaborative recommendations. The explanations did not always increase system \emph{trust}, but could give an indication of recommender system bias, as poor recommendations were often not connected to the user's top neighbours. Finally, the explanation system may provide a starting point for further data exploration.

The objective that was not met, was to enable users to actively steer the recommendation process. This is due to the fact that the \emph{Last.fm API} did not support this functionality. Of course an alternative could have been to make use of other methods in the API to construct our own custom recommender system, but we have chosen to explain the artist recommendations made by the actual recommender instead. Another possibility could have been to use another recommender system altogether, but from the systems that were investigated, e.g. \emph{Spotify}, \emph{Grooveshark}, \emph{Bandcamp}, no significant additional functionality was discovered that could have overcome these issues.

% waarom Last.fm


\section{Future work}

\subsection{Issues}

Future work may include addressing problems with visual clutter, and slow data load as listed in table \ref{table:iteration4:issues}.


\subsection{Evaluation}

For future user tests, \emph{Last.fm} users could be given a pre-test questionnaire to evaluate the Last.fm recommender and its explanations. Such a benchmark could have proven useful in understanding the usefulness of the application. Other evaluation methods that can be used, are for example expert-based evaluation, and heuristic approaches.


\subsection{Visualization and music}

The focus of the literature study was mainly on providing a context for the elements that were used in the application. To improve the initial design, it might have been better to also incorporate some sort of comparative study of visualization techniques for music, especially if we were to built an explanation system for content-based recommendation. On the other hand, this subject may provide enough content for another thesis.


\subsection{Extensions}

The interactive elements could be enhanced, and the amount of interactive elements increased. For example by allowing interaction with edges, the user could dig deeper into the relationship between artists and the corresponding users.

The visual explanation system could be tested using other data sets and collaborative recommendation systems. The model could be extended for use in a hybrid environment, for example by visualizing also tag-based or other relationships among artists in \emph{Last.fm}.


\section{Personal reflection}

All in all this has been an interesting project. During its course, a lot has been learned and the reasoning on this subject has developed as well. In hindsight there are inevitably things that one would do differently, and this is no exception.

\subsection{An overview of how the project unfolded}

The first months of this project were probably the most difficult ones. It was not always easy to determine what to look for. A lot of papers had to be read again in a later stage, since a lot of details were overlooked due to a lack of context and direction. This is probably typical of students coming from the programme \emph{Schakelprogramma Master Toegepaste Informatica}\footnote{\url{http://onderwijsaanbod.kuleuven.be/opleidingen/n/SC_50527959.htm}}. Especially in a one year programme, some classes that could have provided background for the thesis subject, may come late in the academic year, such as the course \emph{Gebruikersinterfaces}\footnote{\url{http://onderwijsaanbod.kuleuven.be/2012/syllabi/n/H04I2AN.htm}}. As a result, this may have affected the motivation for working on the thesis by the end of the first semester.

During the Christmas break, some papers were reread and a better idea of what needed to be done was formed. As a result, the slow progress in the first semester had to be made undone in the second one. Still, looking back, it is not easy to counter this problem, which is perhaps part of the insight gaining process decribed in this thesis.


\subsection{Lessons learned}

Some things that could have been done differently are probably to have started earlier with user tests. The idea explained in this thesis was developed early on in the project, but was evaluated much later. Conducting user studies early on would have yielded more test results, and provided additional experience. Some issues with the application and testing methods are likely to have been discovered at a much earlier stage as well. One of the reasons for stalling, was lack of confidence in the idea, and also a lack of experience in conducting user studies.

%Other than that, reading and writing scientific literature have become a bit more fluent.

%For example the course \emph{Gebruikersinterfaces}\footnote{\url{http://onderwijsaanbod.kuleuven.be/2012/syllabi/n/H04I2AN.htm}} teaches a methodology for conducting user studies.


%\subsection{Results}
% personal score for the result?




%\subsection{Remarks}





% CONTENTS :
% Geef een overzicht van het door jou geleverde werk. Zorg dat het duidelijk is wat je eigen inbreng is en wat je elders gevonden hebt.
% Vergelijk de oorspronkelijke doelstelling met wat je bereikt hebt.
% Vermeld de belangrijkste problemen die je had bij het verwezenlijken van die doelstellingen.
% Wees kritisch en geef de voor- en nadelen van jouw oplossing en vergelijk je bekomen resultaat met beschikbare alternatieven.
% Geef aan welke uitbreidingen en verfijningen je nog zou kunnen/willen doen als je er de tijd voor had.
% 	FUTURE WORK
% 		visualization of content-based recommendation:
%			distance functions of feature vectors in a spatial map
%				...
% 		using this thesis as a starting point with visualization, sensemaking, etc. in literature study
%			using the whitebox in another context than music recommendation











% ------------------------- Bibliography --------------------------
% 
% -----------------------------------------------------------------
%% A small distance to the other stuff in the table of contents (toc)
\addtocontents{toc}{\protect\vspace*{\baselineskip}}

%% The Bibliography
%% ----------------
%% ==> You need a file 'literature.bib' for this.
%% ==> You need to run BibTeX for this (Project | Properties... | Uses BibTeX)
\addcontentsline{toc}{chapter}{References} %'Bibliography' into toc
%\nocite{*} %Even non-cited BibTeX-Entries will be shown.
%\bibliographystyle{alpha} %Style of Bibliography: plain / apalike / amsalpha / ...
\bibliographystyle{abbrv}
\bibliography{bib/references}


%% Index
%% -----
%\clearpage
%\addcontentsline{toc}{chapter}{Index}
%\printindex

%% Appendices
%% ----------
\clearpage
\addcontentsline{toc}{chapter}{Appendices}
% Appendices: Hierin worden de delen van het onderzoek opgenomen die essentieel zijn voor het werk, maar die de leesbaarheid van de tekst zouden verlagen bv. door de lengte (wiskundige afleidingen, experimentele data, voorbeelden, figuren, etc.).
\appendix


\chapter{Use cases}\label{appendix:use_cases}

%%%%%%%%%%%%%%%USE CASE 1
\begin{table}[h]
\caption{Use case 1 \textit{Hover item}}
\begin{center}
	\begin{tabular}{ l p{300px} } % l = left-aligned column; p{...px} = paragraph with specified width.
		\hline
		\textbf{Primary actor:}	& Active user \\
		\textbf{Preconditions:}	& The application has access to the active user's profile; \\
														& The visualization has successfully loaded; \\
		\textbf{Basic flow:}	& (1) The user enters the area of an item node in the graph; \\
													& (2) The system highlights the nodes and edges that are directly connected to the target node (popout technique); \\
													& (3) The system highlights icons next to the graph corresponding to neighbours that have the target item in their profile; \\
													& (4) The user exits the node area; \\
													& (5) The system shows the default layout of the graph; \\
		\hline
	\end{tabular}
\end{center}
\label{tab:use_case1}
\end{table}



%%%%%%%%%%%%%%%USE CASE 2
\begin{table}[h]
\caption{Use case 2 \textit{Hover neighbour}}
\begin{center}
	\begin{tabular}{ l p{300px} }
		\hline
		\textbf{Primary actor:}	& Active user \\
		\textbf{Preconditions:}	& The application has access to the active user's profile; \\
														& The visualization has successfully loaded; \\
		\textbf{Basic flow:}	& (1) The user enters the area of a neighbour icon next to the graph; \\
													& (2) The system highlights the neighbour's icon; \\
													& (3) The system highlights the nodes and edges that connect items that are in the profile of the selected user; \\
													& (4) The user exits the icon area; \\
													& (5) The system shows the default layout of the graph; \\
		\hline
	\end{tabular}
\end{center}
\label{tab:use_case2}
\end{table}



%%%%%%%%%%%%%%%USE CASE 3
\begin{table}[h]
\caption{Use case 3 \textit{Click item}}
\begin{center}
	\begin{tabular}{ l p{300px} }
		\hline
		\textbf{Primary actor:}	& Active user \\
		
		\textbf{Preconditions:}	& The application has access to the active user's profile; \\
														& The visualization has successfully loaded; \\
		
		\textbf{Basic flow:}	& (1) The user clicks an item node in the graph; \\
													& (2) The system highlights the nodes and edges that are directly connected to the target node; \\
													& (3) The system highlights icons next to the graph corresponding to neighbours that have the target item in their profile; \\
													& (5) The system displays additional information about the item and options in an area next to the visualization; information includes a brief introductory text and top tracks; options include the possibility to add the item to the active user's profile. \\
		
		\textbf{Alternate flow:}	& (2.a) the item was already selected: the item is now deselected; \\
		
		\hline
	\end{tabular}
\end{center}
\label{tab:use_case3}
\end{table}


%%%%%%%%%%%%%%%USE CASE 4
\begin{table}[h]
\caption{Use case 4 \textit{Click neighbour}}
\begin{center}
	\begin{tabular}{ l p{300px} }
		\hline
		\textbf{Primary actor:}	& Active user \\
		
		\textbf{Preconditions:}	& The application has access to the active user's profile; \\
														& The visualization has successfully loaded; \\
		
		\textbf{Basic flow:}	& (1) The user clicks an item node in the graph; \\
													& (2) The system highlights the nodes and edges that are directly connected to the target node; \\
													& (3) The system highlights icons next to the graph corresponding to neighbours that have the target item in their profile; \\
													& (4) The system displays additional information about the item and options in an area next to the visualization; information includes a brief introductory text and top tracks; options include the possibility to add the item to the active user's profile. \\
		
		\textbf{Alternate flow:}	& (2.a) the neighbour was already selected: the neighbour is now deselected; \\
		
		\hline
	\end{tabular}
\end{center}
\label{tab:use_case4}
\end{table}


% Questions for the user tests for each iteration
\chapter{Task lists for the user tests}\label{appendix:tasklists}

\section{Task list 1: testing insight and usability}\label{appendix:tasklists:prototype1}

First the user is given some context, i.e., the user knows he/she is using a recommender system to find new music and he/she has a number artists in his/her artist library. Next tasks \ref{table:task:t1}, \ref{table:task:t2} and \ref{table:task:t3} are performed.

\begin{table}
	\caption{Task 1.1: hypothesis generation, no interaction allowed.}
	\begin{tabular}{ p{80px} | p{350px} }
		\hline
		\textbf{Goal/Output}			& Getting an idea of the user's mental model about the visualization when he/she is not allowed to interact with it. \\
		\textbf{Inputs}						& The user has an account, and has built up some listening history. \\
		\textbf{Assumptions}			& The user is logged in. The data has loaded. \\
		\textbf{Steps}						& The user will try to get an overview of the displayed data. Through eye-movements he/she will explore the visualization. The user forms a hypothesis on the visualization rationale. \\
		\textbf{Estimated time} 	& $5$ to $10$ minutes. \\
		\textbf{Instructions}			&
		
		Answer the following questions without interacting with the visualization:
		\begin{enumerate}
			\item Describe what you see. Which visual elements stand out? Which general structures can be identified?
			\item What do you think the visualization does?
			\item Which the elements of the user interface, do you think allow interaction?
			\item What do you think will happen when you:
				\begin{itemize}
					\item hover over an node of the graph?
					\item hover over one of the users?
					\item click on an item?
					\item click on a user?
				\end{itemize}
		\end{enumerate}
		\\
		\hline
	\end{tabular}
	\label{table:task:t1}
\end{table}

\begin{table}
	\caption{Task 1.2: Further familiarization, interaction allowed.}
	\begin{tabular}{ p{80px} | p{350px} }
		\hline
		\textbf{Goal/Output}			& Getting an idea of the user's mental model about the visualization. \\
		\textbf{Inputs}						& See table \ref{table:task:t1}. \\
		\textbf{Assumptions}			& See table \ref{table:task:t1}. \\
		\textbf{Steps}						& The user verifies his/her initial mental model through interactions with the visualization. If the initial mental model is not confirmed, it is adjusted. \\
		\textbf{Estimated time} 	& \\
		\textbf{Instructions}			&
		Try to interact with the visualization. Answer the following questions:
		\begin{enumerate}
			\item Which of the artists displayed in the graph are artist suggestions?
			\item What are the links or edges in the visualization?
			\item	Suppose you want to add an item to your profile, what steps would you undertake?
		\end{enumerate}
		\\
		\hline
	\end{tabular}
	\label{table:task:t2}
\end{table}


\begin{table}
	\caption{Task 1.3: Adding an artist to the music library and motivating the choice(s) made.}
	\begin{tabular}{ p{80px} | p{350px} }
		\hline
		\textbf{Goal/Output}			& \\
		\textbf{Inputs}						& See table \ref{table:task:t1}. \\
		\textbf{Assumptions}			& See table \ref{table:task:t1}. \\
		\textbf{Steps}						& The user clicks an artist of his/her choice. The user clicks \emph{Add to library} and confirms his/her action. The item is added to the profile and the visualization refreshes. \\
		\textbf{Estimated time} 	& $1$ to $5$ minutes. \\
		\textbf{Instructions}			&
		Add an item to your profile. Answer the following questions:
		\begin{enumerate}
			\item Why did you choose that particular item?
			\item Can you give any other reasons why you should pick this item?
			\item Can you give reasons for choosing one of the other items?
		\end{enumerate}
		\\
		\hline
	\end{tabular}
	\label{table:task:t3}
\end{table}



\section{Task list 2: testing the first version of the settings menu}\label{appendix:tasklists:prototype2}

Tables \ref{table:task:t4}, \ref{table:task:t5}, and \ref{table:task:t6} give an overview of the tasks used in the user study to evaluate the settings menu.

\begin{table}
	\caption{Task 2.1: Change the number of shown recommendations up to $20$.}
	\begin{tabular}{ p{80px} | p{350px} }
		\hline
		\textbf{Goal/Output}			& The number of displayed recommendations in the graph has changed. \\
		\textbf{Inputs}						& The user has a \emph{Last.fm} account. The user has authorized the application. \\
		\textbf{Assumptions}			& The user is logged in. The user has navigated to the recommendations page and the vsiualization has loaded. \\
		\textbf{Steps}						& Click the \emph{Settings} button and alter the slider for the number of recommendations. Click \emph{Save}. \\
		\textbf{Estimated time} 	& Less than a minute. \\
		\textbf{Instructions}			& Change the number of shown recommendations up to $20$. \\
		\hline
	\end{tabular}
	\label{table:task:t4}
\end{table}


\begin{table}
	\caption{Task 2.2: Change the threshold to $0.3$.}
	\begin{tabular}{ p{80px} | p{350px} }
		\hline
		\textbf{Goal/Output}			& The threshold used to cluster items has changed, which will affect the connectivity of the graph. \\
		\textbf{Inputs}						& See table \ref{table:task:t4}. \\
		\textbf{Assumptions}			& See table \ref{table:task:t4}. \\
		\textbf{Steps}						& Click the \emph{Settings} button and alter the slider for the threshold. Click \emph{Save}. \\
		\textbf{Estimated time} 	& Less than a minute. \\
		\textbf{Instructions}			& Change the threshold to $0.3$. \\
		\hline
	\end{tabular}
	\label{table:task:t5}
\end{table}



\begin{table}
	\caption{Task 2.3: Change the colours to an encoding that you like.}
	\begin{tabular}{ p{80px} | p{350px} }
		\hline
		\textbf{Goal/Output}			& The colour encodings for hover and click actions has changed, as well as the colour of the active user profile. \\
		\textbf{Inputs}						& See table \ref{table:task:t4}. \\
		\textbf{Assumptions}			& See table \ref{table:task:t4}. \\
		\textbf{Steps}						& Click the \emph{Settings} button and alter the the colour settings. Click \emph{Save}. \\
		\textbf{Estimated time} 	& About a minute or more. \\
		\textbf{Instructions}			& Change the colours to an encoding that you like. \\
		\hline
	\end{tabular}
	\label{table:task:t6}
\end{table}







\section{Task list 3: testing the performance of the evaluation system}\label{appendix:tasklists:prototype3}

Tasks \ref{table:task:t7}, \ref{table:task:t8} \ref{table:task:t9}, and \ref{table:task:t10} are used to further evaluate the explanation system properties.

\begin{table}
	\caption{Task 3.1: Find three neighbours that are closely related to you, based on the visualization.}
	\begin{tabular}{ p{80px} | p{350px} }
		\hline
		\textbf{Goal/Output}			& The user can find three closely related neighbours and can give an adequate motivation for his/her choice. \\
		\textbf{Inputs}						& See table \ref{table:task:t4}. \\
		\textbf{Assumptions}			& See table \ref{table:task:t4}. \\
		\textbf{Steps}						& Based on the visual thinking algorithm in table \ref{table:visual_thinking_algorithm}. \\
		\textbf{Estimated time} 	& $3$ minutes or more, depending on past experience. \\
		\textbf{Instructions}			&
		Find three neighbours that are closely related to you, based on the visualization. Explain why these are closer neighbours than others neighbours on the graph.
		\\
		\hline
	\end{tabular}
	\label{table:task:t7}
\end{table}


\begin{table}
	\caption{Task 3.2: Find three recommended artists you think are interesting.}
	\begin{tabular}{ p{80px} | p{350px} }
		\hline
		\textbf{Goal/Output}			& The user can find three interesting artist recommendations and give an adequate motivation for his/her choice. \\
		\textbf{Inputs}						& See table \ref{table:task:t4}. \\
		\textbf{Assumptions}			& See table \ref{table:task:t4}. \\
		\textbf{Steps}						& Based on the visual thinking algorithm in table \ref{table:visual_thinking_algorithm}. \\
		\textbf{Estimated time} 	& $3$ minutes or more, depending on past experience. \\
		\textbf{Instructions}			&
		Find three recommended artists you think are interesting. Explain why these artists are more interesting than other recommendations shown in the graph.
		\\
		\hline
	\end{tabular}
	\label{table:task:t8}
\end{table}


\begin{table}
	\caption{Task 3.3: Explain the recommendation rationale (transparency).}
	\begin{tabular}{ p{80px} | p{350px} }
		\hline
		\textbf{Goal/Output}			& The high level algorithm for collaborative filtering. \\
		\textbf{Inputs}						& See table \ref{table:task:t4}. \\
		\textbf{Assumptions}			& See table \ref{table:task:t4}. \\
		\textbf{Steps}						& Based on the visual thinking algorithm in table \ref{table:visual_thinking_algorithm}. \\
		\textbf{Estimated time} 	& $3$ minutes or more, depending on past experience. \\
		\textbf{Instructions}			&
		Explain the recommendation rationale. How do you think \emph{Last.fm}'s recommender system works?
		\\
		\hline
	\end{tabular}
	\label{table:task:t9}
\end{table}


\begin{table}
	\caption{Task 3.4: Find a suggestion for an artist you didn't know about.}
	\begin{tabular}{ p{80px} | p{350px} }
		\hline
		\textbf{Goal/Output}			& A new suggestion. \\
		\textbf{Inputs}						& See table \ref{table:task:t4}. \\
		\textbf{Assumptions}			& See table \ref{table:task:t4}. \\
		\textbf{Steps}						& The user looks at each of the suggestions and points out those that are new to him/her. \\
		\textbf{Estimated time} 	& About a minute to find new suggestions. Investigating interesting recommendations and answering the questions may take up up to $15$ minutes. \\
		\textbf{Instructions}			&
		Find a suggestion for an artist you didn't know about, and answer the following questions:
		\begin{itemize}
			\item Would you like to check our this artist's profile and listen one or more songs by this artist (persuasion)?
			\item Do you think the recommender system has made a good suggestion? Would you add it your profile (effectiveness)?
			\item How does it affect your trust in the recommender system (trust)?
		\end{itemize}
		\\
		\hline
	\end{tabular}
	\label{table:task:t10}
\end{table}




% The statistics of the social media usage
\chapter{Quantified self}\label{appendix:statistics}





\chapter{Scientific article}
\includepdf[pages={1-9}]{pdf/scientific_article}

\chapter{Poster}
\includepdf[pages={1}]{pdf/poster}

\chapter{Source code}

The source code can be found on the disk added to this text.



\newpage
% ----------------------- Back cover ------------------------------
% Please fill in:
% - Department
% - Department's address
% - Telephone number and fax number
% - e-mail
% -----------------------------------------------------------------
\thispagestyle{empty}
\sffamily
%
\begin{textblock}{191}(113,-11)
{\color{blueline}\rule{160pt}{5.5pt}}
\end{textblock}
%
\begin{textblock}{191}(168,-11)
{\color{blueline}\rule{5.5pt}{59pt}}
\end{textblock}
%
\begin{textblock}{183}(-24,-11)
\textblockcolour{}
\flushright
\fontsize{7}{7.5}\selectfont
\textbf{AFDELING}\\
Straat nr bus 0000\\
3000 LEUVEN, BELGI\"{E}\\
tel. + 32 16 00 00 00\\
fax + 32 16 00 00 00\\
@kuleuven.be\\
www.kuleuven.be\\
\end{textblock}
%
\begin{textblock}{191}(154,-7)
\textblockcolour{}
\includegraphics*[height=16.5truemm]{sedes}
\end{textblock}
%
\begin{textblock}{191}(-20,235)
{\color{bluetitle}\rule{544pt}{55pt}}
\end{textblock}
\end{document}
